\chapter{The Standard Model of HF}

Gödel's first incompleteness theorem states that, if HF is \textit{consistent}, 
then there is a sentence $\delta$ such that neither $\vdash \delta$ nor
$\vdash \neg\delta$. According to soundness, 
this implies there exists model of HF which is
a model of $\delta$ and there exists another model of HF which doesn't model $\delta$.

For comparison, in Peano Arithmetic (PA), the 'Gödel sentence' is true in the standard model of 
the natural numbers but false in some "non-standard" model of PA. 
Similarly, in HF, the sentence $\delta$ is true in the \textit{standard model of HF}, 
which we will construct in this chapter.

We will start by establishing some properties of constant terms.

\section{Constant terms}

\begin{definition}[Constant term]
    \lean{HF.C, HF.C.length}
    \label{def:C+C.length}
    \leanok
    \uses{def:Lang}
    The class of \textit{constant terms}, denoted by $\mathbb{C}$, is the class of terms which
    contain no variables, i.e. $\mathbb{C}$ is the smallest class of terms such that 
    $\varnothing \in \mathbb{C}$ and 
    $\sigma \lhd \tau \in \mathbb{C}$ for any $\sigma, \tau \in \mathbb{C}$.

    For any $\tau \in \mathbb{C}$, the number of appearances of $\lhd$ in $\tau$ is called the
    \textit{length} of $\tau$, denoted by $l(\tau)$.
\end{definition}

\begin{lemma}
    \lean{HF.C.exists_finset_shorter_and_mem_iff_iSup}
    \label{lem:C.exists_finset_shorter_and_mem_iff_iSup}
    \leanok
    \uses{def:prf, thm:completeness, lem:notin_empty+mem_enlarge+mem_enlarge_empty, def:C+C.length}
    For every non-empty $\tau \in \mathbb{C}$, there are finitely many 
    $\tau_1, \ldots, \tau_m \in \mathbb{C}$, all shorter than $\tau$, such that
    $$
    \vdash (x \in \tau \leftrightarrow x = \tau_1 \lor \ldots \lor x = \tau_m).
    $$
\end{lemma}

\begin{proof}
    \leanok
    We apply induction on the length $l(\tau)$.
    The case $l(\tau) = 0$, i.e. $\tau = \varnothing$, is trivial.
    Now assume that the lemma has been established for all non-zero constant terms shorter 
    than $\tau$, and let $\tau$ be $\sigma \lhd \mu$.
    If $\sigma = \varnothing$, then $\tau = \varnothing \lhd \mu$, 
    where $\mu$ is shorter than $\tau$.
    We have the theorem of HF
    $$
    \vdash (x \in \varnothing \lhd \mu \leftrightarrow x \in \varnothing \lor x=\mu
    \leftrightarrow x=\mu),
    $$
    i.e. $m=1$ and $\tau_1 = \mu$. 
   
    Now suppose $\sigma \neq \varnothing$. As $\sigma$ is shorter than $\tau$,
    there are $\sigma_1, \ldots, \sigma_n \in \mathbb{C}$ such that
    $$
    \vdash (x \in \sigma \leftrightarrow x=\sigma_1 \lor \ldots \lor x=\sigma_n)
    $$
    Since $\vdash (x \in \tau \leftrightarrow x \in \sigma \lor x=\mu)$, we get
    $$
    \vdash (x \in \tau \leftrightarrow x=\sigma_1 \lor \ldots \lor x=\sigma_n \lor x=\mu).
    $$
    By inductive assumption, 
    each $\sigma_i$ is shorter than $\sigma$, thus also shorter than $\tau$. 
    Also, $\mu$ is shorter than $\tau$.
\end{proof}

\begin{lemma}
    \lean{HF.C.exists_mem_and_notin_of_not_eq}
    \label{lem:C.exists_mem_and_notin_of_not_eq}
    \leanok
    \uses{def:prf, thm:completeness, thm:exten_prop, def:C+C.length, 
        lem:C.exists_finset_shorter_and_mem_iff_iSup}
    Let $\sigma, \tau \in \mathbb{C}$ be such that $\not\vdash \sigma = \tau$. 
    Then, for some $\nu \in \mathbb{C}$, one of the two possibilities occurs:
    \begin{enumerate}
        \item $\vdash \nu \in \sigma \quad \text{and} \quad \vdash \nu \notin \tau$.
        \item $\vdash \nu \notin \sigma \quad \text{and} \quad \vdash \nu \in \tau$.
    \end{enumerate}
\end{lemma}

\begin{proof}
    First, consider the case where $\sigma = \varnothing$ or $\tau = \varnothing$.
    Suppose $\sigma = \varnothing$.
    Then $\not\vdash \sigma = \tau$ implies $\tau \neq \varnothing$, and thus
    $\tau = \mu \lhd \nu$ for some constant terms $\mu, \nu$.
    It follows that $\vdash \nu \in \tau$ and $\vdash \nu \notin \sigma$, 
    and thus possibility 1. holds.
    Analogously, if $\tau = \varnothing$, possibility 2. holds.

    Now, we apply induction on the length $l(\sigma)$.
    The case $l(\sigma) = 0$ has been considered already.
    For $l(\sigma) > 0$, suppose the lemma has been established for all non-zero constant terms 
    shorter than $\sigma$, and assume $\tau$ is such that $\not\vdash \sigma = \tau$.
    The case $\tau = \varnothing$ has been dealt with.
    It remains to consider the case where neither $\sigma$ nor $\tau$ is $\varnothing$.
    By Lemma \ref{lem:C.exists_finset_shorter_and_mem_iff_iSup}, there are finitely many
    $\sigma_1, \ldots, \sigma_m \in \mathbb{C}$, all shorter than $\sigma$, and finitely many
    $\tau_1, \ldots, \tau_n \in \mathbb{C}$, all shorter than $\tau$, such that
    $$
    \vdash (x \in \sigma \leftrightarrow x = \sigma_1 \lor \ldots \lor x = \sigma_m)
    \quad \text{and} \quad 
    \vdash (x \in \tau \leftrightarrow x = \tau_1 \lor \ldots \lor x = \tau_n).
    $$
    If for every $\sigma_i$ there exists exactly one $\tau_j$ such that $\vdash \sigma_i = \tau_j$,
    then $\vdash (x \in \sigma \leftrightarrow x \in \tau)$, and thus, 
    by Theorem \ref{thm:exten_prop}, $\vdash \sigma = \tau$, a contradiction.
    Hence, either there is some $\sigma_i$ such that for every $\tau_j$ we have
    $\not\vdash \sigma_i = \tau_j$, or there is some $\tau_j$ such that for every $\sigma_i$ we have
    $\not\vdash \sigma_i = \tau_j$.

    Suppose the first case.
    Since this $\sigma_i$ is shorter than $\sigma$, by inductive assumption, 
    either possibility 1. or 2. holds for $\sigma_i$ and every $\tau_j$, for suitable $\nu_j$.
    Thus, by Theorem \ref{thm:exten_prop}, $\vdash \sigma_i \neq \tau_j$ for all $j$.
    We now substitute $\sigma_i$ for $x$ in 
    $$
    \vdash (x \notin \tau \leftrightarrow x \neq \tau_1 \land \ldots \land x \neq \tau_n),
    $$
    which implies $\vdash \sigma_i \notin \tau$.
    But $\vdash \sigma_i \in \sigma$, so possibility 1. holds with $\nu = \sigma_i$.
    The second case where there exists some $\tau_j$ such that for every $\sigma_i$ we have
    $\not\vdash \sigma_i = \tau_j$ is analogous and implies possibility 2.
\end{proof}

\begin{corollary}
    \lean{HF.C.ne_of_not_eq}
    \label{cor:C.ne_of_not_eq}
    \leanok
    \uses{def:prf, thm:completeness, thm:exten_prop, def:C+C.length, 
        lem:C.exists_mem_and_notin_of_not_eq}
    For every $\sigma, \tau \in \mathbb{C}$, $\not\vdash \sigma = \tau$ implies 
    $\vdash \sigma \neq \tau$.
\end{corollary}

\begin{proof}
    \leanok
    Follows directly from Lemma \ref{lem:C.exists_mem_and_notin_of_not_eq} and 
    Theorem \ref{thm:exten_prop}.
\end{proof}

\begin{corollary}
    \lean{HF.C.eq_of_forall_mem_iff_mem}
    \label{cor:C.eq_of_forall_mem_iff_mem}
    \uses{def:prf, def:C+C.length, lem:C.exists_mem_and_notin_of_not_eq}
    \leanok
    If $\sigma, \tau \in \mathbb{C}$ are such that $\vdash \nu \in \sigma$ if and only if
    $\vdash \nu \in \tau$ for every $\nu \in \mathbb{C}$, and HF is consistent, 
    then $\vdash \sigma = \tau$.
\end{corollary}

\begin{proof}
    \leanok
    Follows directly from Lemma \ref{lem:C.exists_mem_and_notin_of_not_eq} 
    using proof by contradiction.
\end{proof}

\section{The model}

\begin{definition}[The standard model of HF]
    \label{def:stdModel}
    \lean{HF.stdModel, HF.stdModel.struc}
    \uses{def:Lang, def:C+C.length}
    \leanok
    Introduce the equivalence relation $\approx$ on $\mathbb{C}$ given by
    $$
    \sigma \approx \tau \,\text{ if and only if } \vdash \sigma = \tau.
    $$
    Then the $\mathcal{L}$-structure 
    $\mathfrak{S} = \langle\mathbb{S}, \varnothing, \in, \lhd \rangle$ is defined by the following:
    \begin{enumerate}
        \item $\mathbb{S} = \mathbb{C}/{\approx}$.
        \item The abbreviation $\varnothing$ for $[\varnothing]$.
        \item The binary relation $\in$ on $\mathbb{S}$ defined by
                $$[\sigma] \in [\tau] \leftrightarrow \,\vdash \sigma \in \tau.$$
        \item The binary operation $\lhd$ on $\mathbb{S}$ defined by
                $$[\sigma] \lhd [\tau] = [\sigma \lhd \tau].$$
    \end{enumerate}
\end{definition}

\begin{theorem}
    \label{thm:stdModel.model_of_consistent}
    \lean{HF.stdModel.model_of_consistent}
    \uses{def:Lang, def:Axioms, thm:completeness, lem:notin_empty+mem_enlarge+mem_enlarge_empty, 
        thm:exten_prop, cor:C.ne_of_not_eq, cor:C.eq_of_forall_mem_iff_mem, def:stdModel}
    \leanok
    If HF is consistent, then $\mathfrak{S}$ is a model of HF.
\end{theorem}

\begin{proof}
    We check that the axioms of HF are modelled by $\mathfrak{S}$.

    \textit{Axiom 1.} Suppose $[\sigma] \in \mathbb{S}$ is such that 
    $\mathfrak{S} \vDash [\tau] \notin [\sigma]$ for every constant term $\tau$.
    Then $\sigma$ cannot be of the form $\mu \lhd \nu$ for some constant terms $\mu, \nu$,
    as otherwise $\mathfrak{S} \vDash [\nu] \in [\sigma]$ by \ref{def:stdModel}.3.
    Thus, $\sigma = \varnothing$.
    Conversely, we have $\mathfrak{S} \vDash [\tau] \notin [\varnothing]$ for all $\tau$,
    as, by \ref{def:stdModel}.3, $\mathfrak{S} \vDash [\tau] \in [\varnothing]$ if and only if
    $\vdash \tau \in \varnothing$, and the latter contradicts 
    Lemma \ref{lem:notin_empty+mem_enlarge+mem_enlarge_empty} and the assumption of consistency.

    \textit{Axiom 2.} Theorem \ref{thm:exten_prop} holds for $\mathfrak{S}$, by 
    Corollary \ref{cor:C.eq_of_forall_mem_iff_mem}.
    Thus, Axiom 2 (Definition \ref{def:Axioms}.2) will be modelled by $\mathfrak{S}$ if 
    for every $\tau,\mu,\nu \in \mathbb{C}$, 
    $$
    \vdash \tau \in \mu \lhd \nu \,\text{ if and only if } 
    (\vdash \tau \in \mu \,\text{ or } \vdash \tau = \nu).
    $$
    First, assume $\vdash \tau \in \mu \lhd \nu$. 
    If $\vdash \tau = \nu$, then the desired implication holds immediately.
    If $\not\vdash \tau = \nu$, then, by Corollary \ref{cor:C.ne_of_not_eq},
    $\vdash \tau \neq \nu$.
    So, in this case, $\vdash \tau \in \mu \lhd \nu$ implies
    $\vdash (\tau \in \mu \lor \tau = \nu) \land \tau \neq \nu$, and thus
    $\vdash \tau \in \mu$. Conversely, 
    $(\vdash \tau \in \mu \,\text{ or } \vdash \tau = \nu)$ implies
    $\vdash (\tau \in \mu \lor \tau = \nu)$, and thus
    $\vdash \tau \in \mu \lhd \nu$.
    
    \textit{Axiom 3.} For each equivalence class $a \in \mathbb{S}$, let $h(a)$ be the smallest
    length $l(\tau)$ such that $\tau \in a$.
    Then, for every $a \neq \varnothing$, there exist $b,c \in \mathbb{S}$ such that 
    $a = b \lhd c$ holds true in $\mathbb{S}$, and both $h(b) < h(a)$ and $h(c) < h(a)$.
    Given an $\mathcal{L}$-formula $\alpha(x)$ with one free variable $x$,
    define 
    $$
    A_\alpha = \{a \in \mathbb{S} : \alpha (a) \text{ holds true in $\mathbb{S}$}\}.
    $$
    Now, assume $\varnothing \in A_\alpha$ and that $a,b \in A_\alpha$ implies 
    $a \lhd b \in A_\alpha$ for all $a,b \in \mathbb{S}$.
    Then, by induction on $h(c)$, it follows that $c \in A_\alpha$ for all $c \in \mathbb{S}$,
    and thus $\alpha(c)$ holds true for all $c \in \mathbb{S}$.
    This establishes Axiom 3.
\end{proof}