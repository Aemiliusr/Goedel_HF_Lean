\chapter{The Standard Model of HF}

Gödel's first incompleteness theorem states that, if HF is \textit{consistent}, i.e.
$\not\vdash\bot$, then there is a sentence $\delta$ such that neither $\vdash \delta$ nor
$\vdash \neg\delta$. According to the completeness theorem, 
this implies that there is one model of HF which is
a model of $\delta$ and another model of HF which does not model $\delta$.

For comparison, in Peano Arithmetic (PA), the 'Gödel sentence' is true in the standard model of 
the natural numbers but false in some "non-standard" model of PA. 
Similarly, in HF, the sentence $\delta$ is true in the \textit{standard model of HF}, 
which we will construct in this chapter.

We will start by establishing some properties of constant terms.

\section{Constant terms}

\begin{definition}[Constant term]
    \lean{HF.C}
    \label{def:C}
    \uses{def:Lang}
    The class of \textit{constant terms}, denoted by $\mathbb{C}$, is the class of terms which
    contain no variables, i.e. $\mathbb{C}$ is the smallest class of terms such that 
    $\varnothing \in \mathbb{C}$ and 
    $\sigma \lhd \tau \in \mathbb{C}$ for any $\sigma, \tau \in \mathbb{C}$.

    For any $\tau \in \mathbb{C}$, the number of appearances of $\lhd$ in $\tau$ is called the
    \textit{length} of $\tau$, denoted by $l(\tau)$.
\end{definition}

\begin{lemma}
    For every non-empty $\tau \in \mathbb{C}$, there are finitely many 
    $\tau_1, \ldots, \tau_m \in \mathbb{C}$, all shorter than $\tau$, such that
    $$
    \vdash (x \in \tau \leftrightarrow x = \tau_1 \lor \ldots \lor x = \tau_m).
    $$
\end{lemma}

\begin{proof}
    The shortest non-empty $\tau$ is $\varnothing \lhd \varnothing$. 
    We have the theorem of HF
    $$
    \vdash (x \in \varnothing \lhd \varnothing \leftrightarrow x \in \varnothing \lor x=\varnothing
    \leftrightarrow x=\varnothing),
    $$
    i.e., $m=1$ and $\tau_1 = \varnothing$. 
    Now assume that the lemma has been established for all non-zero constant terms shorter 
    than $\tau$, and let $\tau$ be $\sigma \lhd \mu$. 
    Then $\sigma$ is shorter than $\tau$, 
    hence there are $\sigma_1, \ldots, \sigma_n \in \mathbb{C}$ such that
    $$
    \vdash (x \in \sigma \leftrightarrow x=\sigma_1 \lor \ldots \lor x=\sigma_n)
    $$
    Since $\vdash (x \in \tau \leftrightarrow x \in \sigma \lor x=\mu)$, we get
    $$
    \vdash (x \in \tau \leftrightarrow x=\sigma_1 \lor \ldots \lor x=\sigma_n \lor x=\mu).
    $$
    By inductive assumption, 
    each $\sigma_i$ is shorter than $\sigma$, thus also shorter than $\tau$. 
    Also $\mu$ is shorter than $\tau$.
\end{proof}

\begin{lemma}
    Let $\sigma, \tau \in \mathbb{C}$ be such that $\not\vdash \sigma = \tau$. 
    Then, for some $\nu \in \mathbb{C}$, one of the two possibilities occurs:
    \begin{enumerate}
        \item $\vdash \nu \in \sigma \quad \text{and} \quad \nu \notin \tau$.
        \item $\vdash \nu \notin \sigma \quad \text{and} \quad \nu \in \tau$.
    \end{enumerate}
\end{lemma}

\begin{proof}
    Under construction.
\end{proof}

\begin{corollary}
    For every $\sigma, \tau \in \mathbb{C}$, $\not\vdash \sigma = \tau$ implies 
    $\vdash \sigma \neq \tau$.
\end{corollary}

\begin{corollary}
    If $\sigma, \tau \in \mathbb{C}$ are such that $\vdash \nu \in \sigma$ if and only if
    $\vdash \nu \in \tau$ for every $\nu \in \mathbb{C}$, then $\vdash \sigma = \tau$.
\end{corollary}

\section{The model}

\begin{definition}[The standard model of HF]
    \label{def:stdModel}
    \lean{HF.stdModel}
    \uses{def:Lang, def:C}
    \leanok
    Introduce the equivalence relation $\equiv$ on $\mathbb{C}$ given by
    $$
    \sigma \equiv \tau \leftrightarrow \,\vdash \sigma = \tau.
    $$
    Then the $\mathcal{L}-$structure 
    $\mathfrak{S} = \langle\mathbb{S}, \varnothing, \in, \lhd \rangle$ is defined by the following:
    \begin{enumerate}
        \item $\mathbb{S} = \mathbb{C}/{\equiv}$.
        \item The abbreviation $\varnothing$ for $[\varnothing]$.
        \item The binary relation $\in$ on $\mathbb{S}$ defined by
                $$[\sigma] \in [\tau] \leftrightarrow \,\vdash \sigma \in \tau.$$
        \item The binary operation $\lhd$ on $\mathbb{S}$ defined by
                $$[\sigma] \lhd [\tau] = [\sigma \lhd \tau].$$
    \end{enumerate}
\end{definition}

\begin{theorem}
    \label{stdModel.model_of_consistent}
    \lean{HF.stdModel.model_of_consistent}
    \uses{def:Lang, def:Axioms, def:stdModel}
    \leanok
    If HF is consistent, then $\mathfrak{S}$ is a model of HF.
\end{theorem}

\begin{proof}
    Under construction.
\end{proof}