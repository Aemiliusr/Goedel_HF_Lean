\chapter{Gödel’s First Incompleteness Theorem}

With the coding system and the provability formula in place, 
we can now prove Gödel’s first incompleteness theorem.
The only ingredient missing is the existence of self-referential sentences, 
which is exactly what Gödel’s diagonal lemma provides.
The latter essentially states that for any $\mathcal{L}$-formula $\alpha$ with one free variable,
there exists a sentence $\delta$ such that $\delta$ is equivalent to $\alpha$ with the code
of $\delta$ substituted for the free variable.
Then, to prove Gödel’s first incompleteness theorem, we apply the diagonal lemma to the negation
of the provability formula, which implies there exists a sentence $\delta$ equivalent to the 
sentence with meaning 'the sentence $\delta$ is not a theorem of HF'.
This yields a contradiction if we assume $\delta$ to be provable or disprovable.

To establish Gödel’s diagonal lemma, we first need to construct, for every variable $x_i$, 
a $p$-function $K$ such that for every $\mathcal{L}$-formula $\varphi(x_i)$,
$$
\vdash K (\ulcorner{\varphi}\urcorner) = \ulcorner{\varphi(\ulcorner {\varphi} \urcorner)}\urcorner.
$$
This is done by letting $K$ be the replacement function, introduced in due course.
However, to prove this property of $K$, we first need a unary $p$-function $H$
that satisfies $H(\mu) = \ulcorner {\mu} \urcorner$ for any code $\mu$, which in turn requires
another unary $p$-function $W$ that is the restriction of $H$ to the codes of variables.

\begin{lemma}
    \label{lem:Code.exists_pFunc_forall_var_eq_code}
    \uses{def:pFunc, def:Ord.pFuncRecursive, def:C+C.length, def:Code.Symbol}
    There exists a unary $p$-function $W$ such that, for any variable $x_i$,
    $$
    \vdash W(\ulcorner{x_i}\urcorner) = \ulcorner{(\ulcorner{x_i}\urcorner)}\urcorner.
    $$
\end{lemma}

\begin{proof}
    Recall that the codes of variables are defined as 
    $\ulcorner{x_1}\urcorner = \varnothing \lhd \varnothing, \quad 
    \ulcorner{x_2}\urcorner = \ulcorner{x_1}\urcorner \lhd \ulcorner{x_1}\urcorner,\quad\ldots\quad,
    \ulcorner{x_{k+1}}\urcorner = \ulcorner{x_k}\urcorner \lhd \ulcorner{x_k}\urcorner$.\\
    Define $W$ recursively on ordinals (see Definition \ref{def:Ord.pFuncRecursive}):
    $$
    W(x) = \begin{cases}
        \varnothing & \text{if } x = \varnothing, \\
        \langle \ulcorner{\lhd}\urcorner, W(\operatorname{pred}(x)), W(\operatorname{pred}(x)) 
        \rangle & \text{otherwise}.
    \end{cases}
    $$ 
    We apply induction on the index of the ordinal $x_i$, i.e. on its length 
    (see Definition \ref{def:C+C.length}).
    First,
    \begin{equation*}
        \begin{split}
    W(\ulcorner(x_1)\urcorner) &= W(\operatorname{succ}(\varnothing)) =
    \langle \ulcorner{\lhd}\urcorner, W(\varnothing), W(\varnothing) \rangle =
    \langle \ulcorner{\lhd}\urcorner, \varnothing, \varnothing \rangle \\ & = 
    \ulcorner{\varnothing \lhd \varnothing}\urcorner = 
    \ulcorner{(\ulcorner{x_1}\urcorner)}\urcorner.
    \end{split}
    \end{equation*}
    Now, suppose that $W(\ulcorner{x_i}\urcorner) = \ulcorner{(\ulcorner{x_i}\urcorner)}\urcorner$.
    Then,
    \begin{equation*}
        \begin{split}
    W(\ulcorner{x_{i+1}}\urcorner) & = W(\operatorname{succ}(\ulcorner{x_i}\urcorner)) =
    \langle \ulcorner{\lhd}\urcorner, W(\ulcorner{x_i}\urcorner), W(\ulcorner{x_i}\urcorner) \rangle
    \\ & = \langle \ulcorner{\lhd}\urcorner, \ulcorner{(\ulcorner{x_i}\urcorner)}\urcorner,
    \ulcorner{(\ulcorner{x_i}\urcorner)}\urcorner \rangle 
    = \ulcorner{(\ulcorner{x_i}\urcorner \lhd \ulcorner{x_i}\urcorner)}\urcorner =
    \ulcorner{(\ulcorner{x_{i+1}}\urcorner)}\urcorner.
        \end{split}
    \end{equation*}
\end{proof}

\begin{lemma}
    \label{lem:Code.exists_pFunc_eq_code}
    \uses{def:singleton+pair, def:pFunc, thm:rank.exists_pFuncRecursive, def:C+C.length, 
    def:Code.Symbol, def:Code.Term, lem:Code.exists_pFunc_forall_var_eq_code}
    There exists a unary $p$-function $H$ such that, for any code $\mu$,
    $$
    \vdash H(\mu) = \ulcorner{\mu}\urcorner.
    $$
\end{lemma}

\begin{proof}
    Recall the definition of an ordered pair:
    \begin{equation*}
        \begin{split}
    \langle x, y \rangle & = \{\{x\}, \{x, y\}\} = 
        \{\varnothing \lhd x, (\varnothing \lhd x) \lhd y\} \\
        & = (\varnothing \lhd (\varnothing \lhd x)) \lhd
        ((\varnothing \lhd x) \lhd y).
        \end{split}
    \end{equation*}
    It follows that, for any terms $\sigma,\tau$,
    \begin{equation*}
        \begin{split}
            \ulcorner {\langle \sigma, \tau \rangle} \urcorner & =
            \ulcorner {(\varnothing \lhd (\varnothing \lhd x)) \lhd
            ((\varnothing \lhd x) \lhd y)} \urcorner \\
            & = \langle \ulcorner{\lhd}\urcorner,
            \langle\ulcorner{\lhd}\urcorner, \varnothing,
            \langle \ulcorner{\lhd}\urcorner, \varnothing, 
            \ulcorner{\sigma}\urcorner \rangle  \rangle,
            \langle \ulcorner{\lhd}\urcorner, 
            \langle \ulcorner{\lhd}\urcorner, \varnothing, 
            \ulcorner{\sigma}\urcorner \rangle, 
            \ulcorner{\tau}\urcorner \rangle \rangle.
        \end{split}
    \end{equation*}
    Let $t(x,y)$ be the term 
    $\langle \ulcorner{\lhd}\urcorner,
    \langle\ulcorner{\lhd}\urcorner, \varnothing,
    \langle \ulcorner{\lhd}\urcorner, \varnothing, x \rangle  \rangle,
    \langle \ulcorner{\lhd}\urcorner, 
    \langle \ulcorner{\lhd}\urcorner, \varnothing, x \rangle, y \rangle \rangle$.
    Then $t(\ulcorner{\sigma}\urcorner, \ulcorner{\tau}\urcorner) = 
    \ulcorner {\langle \sigma, \tau \rangle} \urcorner$ for any terms $\sigma,\tau$.

    Now define $H$ recursively on rank (see Theorem \ref{thm:rank.exists_pFuncRecursive}):
    $$
    H(x) = \begin{cases}
        W(x) & \text{if $x$ is an ordinal}, \\
        t(H(u), H(v)) & \text{if } x = \langle u, v \rangle \text{ for some } u,v, \\
        \varnothing & \text{otherwise}.
    \end{cases}
    $$
    This definition makes sense as no ordered pair is an ordinal 
    ($\varnothing \in x$ for every non-zero ordinal $x$, whereas 
    $\varnothing \notin \langle u, v\rangle$ for all $u,v$), and
    $x = \langle u, v \rangle$ implies $u,v \in \operatorname{cl}(x)$.

    Suppose $\mu \in \Gamma$, i.e. $\mu$ is a code. 
    If $\mu=\varnothing$ or $\mu$ is the code of a variable,
    then $H(\mu) = W(\mu) = \ulcorner{\mu}\urcorner$ by 
    Lemma \ref{lem:Code.exists_pFunc_forall_var_eq_code}.
    All other terms $\mu \in \Gamma$ are built from $\varnothing$ and codes of variables by
    forming ordered pairs.
    Since for $\mu = \langle \sigma, \tau \rangle$ the terms $\sigma,\tau$ are shorter than $\mu$
    (see Definition \ref{def:C+C.length}), we may proceed by induction on the length of $\mu$.
    Assume that $H(\sigma) = \ulcorner{\sigma}\urcorner$ and $H(\tau) = \ulcorner{\tau}\urcorner$.
    Then, by the definition of $t(x,y)$,
    $$
    \ulcorner{\mu}\urcorner = \ulcorner {\langle \sigma, \tau \rangle} \urcorner =
    t(\ulcorner{\sigma}\urcorner, \ulcorner{\tau}\urcorner) = t(H(\sigma), H(\tau)) = H(\mu),
    $$
    Q.E.D.
\end{proof}

\begin{lemma}[The replacement function]
    \label{lem:Code.exists_repl}
    \uses{def:pFunc, def:C+C.length, def:Code.Symbol, def:Code.Term, def:Code.Formula, 
        lem:IsSigma.Relations}
    There exists a ternary $p$-function, denoted by $\operatorname{REPL}$, such that
    $$
    \vdash \operatorname{REPL}(\ulcorner{x_i}\urcorner, \ulcorner{\tau}\urcorner, 
    \ulcorner{\varphi}\urcorner) = \ulcorner{\varphi(x_i/\tau)}\urcorner
    $$
    for every variable $x_i$, term $\tau$ and $\mathcal{L}$-formula $\varphi$, where
    $\varphi(x_i/\tau)$ results from replacing each free occurrence of $x_i$ in $\varphi$ by $\tau$.
\end{lemma}

\begin{proof}
    As the defining $\mathcal{L}$-formula $\varphi(v,t,y,y^\prime)$ for $\operatorname{REPL}$ 
    choose:
    \begin{equation*}
        \begin{split}
    & [(\exists!z)\operatorname{RepVarTermForm}(v,t,y,z) \land 
    \operatorname{RepVarTermForm}(v,t,y,y^\prime)] \lor \\
    & [\not (\exists!z)\operatorname{RepVarTermForm}(v,t,y,z) \land y^\prime = 0].
        \end{split}
    \end{equation*}
    Then it is clear that $(\exists!y^\prime)\varphi(v,t,y,y^\prime)$.
    By the meaning of the $\Sigma$-formula $\operatorname{REPL}$, as described in
    \ref{lem:IsSigma.Relations}, the result follows.
    For the precise details, including the written-out $\Sigma$-formula $\operatorname{REPL}$,
    the reader is referred to Sections 3 and 5 of \cite{swierczkowski2003finite}.
\end{proof}

\begin{lemma}
    \label{lem:Code.exists_pFunc_forall_form_eq_code_form}
    \lean{HF.Code.exists_pFunc_forall_form_eq_code_form}
    \uses{def:pFunc, def:C+C.length, def:Code.Formula, lem:Code.exists_pFunc_eq_code,
        lem:Code.exists_repl}
    \leanok
    For every variable $x_i$, there exists a unary $p$-function $K$ such that, 
    for any $\mathcal{L}$-formula $\varphi(x_i)$,
    $$
    \vdash K (\ulcorner{\varphi}\urcorner) = 
    \ulcorner{\varphi(\ulcorner {\varphi} \urcorner)}\urcorner.
    $$
\end{lemma}

\begin{proof}
    First, note that the composite of $p$-functions is a $p$-function.
    Now put
    $$
    K(x) = \operatorname{REPL}(\ulcorner{x_i}\urcorner, H(x), x).
    $$
    Let $\tau = \ulcorner{\varphi}\urcorner$.
    Then, by \ref{lem:Code.exists_pFunc_eq_code}, $\ulcorner{\tau}\urcorner = H (\tau)$ and
    furthermore, by \ref{lem:Code.exists_repl},
    $$
    \vdash \operatorname{REPL}(\ulcorner{x_i}\urcorner, H(\ulcorner{\varphi}\urcorner), 
    \ulcorner{\varphi}\urcorner) = \ulcorner{\varphi(\ulcorner {\varphi} \urcorner)}\urcorner,
    $$
    i.e. $\vdash K (\ulcorner{\varphi}\urcorner)=
    \ulcorner{\varphi(\ulcorner {\varphi} \urcorner)}\urcorner$.
\end{proof}

\begin{theorem}[Gödel's Diagonal Lemma]
    \label{thm:diagonal}
    \lean{HF.diagonal}
    \uses{lem:Code.exists_pFunc_forall_form_eq_code_form}
    \leanok
    For any $\mathcal{L}$-formula $\alpha$ with one free variable, 
    there exists a sentence $\delta$ such that 
    $$
    \vdash \delta \leftrightarrow \alpha(\ulcorner{\delta}\urcorner).
    $$
\end{theorem}

\begin{proof}
    Substitute $K(x_i)$ for $x_i$ in $\alpha(x_i)$, 
    and denote by $\beta(x_i)$ the resulting $\mathcal{L}$-formula.
    Thus,
    $$
    \vdash \beta(x_i) \leftrightarrow \alpha(K(x_i)),
    $$
    and if we substitute $\ulcorner{\beta}\urcorner$ for $x_i$, 
    by Lemma \ref{lem:Code.exists_pFunc_forall_form_eq_code_form},
    $$
    \vdash \beta(\ulcorner{\beta}\urcorner) \leftrightarrow 
    \alpha(\ulcorner{\beta(\ulcorner{\beta}\urcorner)}\urcorner).
    $$
    Thus, $\delta = \beta(\ulcorner{\beta}\urcorner)$ satisfies the required property.
\end{proof}

\begin{theorem}[Gödel's First Incompleteness Theorem]
    \label{thm:first_incompleteness}
    \lean{HF.first_incompleteness}
    \uses{thm:stdModel.prf_iff_models, thm:prf_iff_prf_Pf_code, thm:diagonal}
    \leanok
    If HF is consistent, then there exists a sentence $\delta$ such that
    neither $\vdash \delta$ nor $\vdash \neg \delta$. 
    Furthermore, $\mathfrak{S} \vDash \delta$.
\end{theorem}

\setcounter{equation}{0}
\begin{proof}
    \leanok
    We apply Theorem \ref{thm:diagonal} to the negation of the provability formula, 
    $\neg\operatorname{Pf}(x)$.
    Then, there exists a sentence $\delta$ such that
    \begin{equation}
        \label{eq:diag}
        \vdash \delta \leftrightarrow \neg \operatorname{Pf}(\ulcorner{\delta}\urcorner).
    \end{equation}
    \begin{itemize}
        \item Suppose $\vdash \delta$. Then, by Theorem \ref{thm:prf_iff_prf_Pf_code},
        $\vdash \operatorname{Pf}(\ulcorner{\delta}\urcorner)$.
        On the other hand, by \ref{eq:diag}, 
        $\vdash \neg \operatorname{Pf}(\ulcorner{\delta}\urcorner)$.
        This contradicts the assumption of consistency.
        \item Suppose $\vdash \neg \delta$. Then, by \ref{eq:diag},
        $\vdash \operatorname{Pf}(\ulcorner{\delta}\urcorner)$, and hence, by
        Theorem \ref{thm:prf_iff_prf_Pf_code}, $\vdash \delta$.
        This contradicts the assumption of consistency.
        \item Suppose $\mathfrak{S} \not\vDash \delta$. 
        Then, as $\delta$ is a sentence, $\mathfrak{S} \vDash \neg\delta$, and hence,
        by \ref{eq:diag}, $\mathfrak{S} \vDash \operatorname{Pf}(\ulcorner{\delta}\urcorner)$.
        As $\operatorname{Pf}(\ulcorner{\delta}\urcorner)$ is a $\Sigma$-sentence, it follows from 
        Theorem \ref{thm:stdModel.prf_iff_models} that
        $\vdash \operatorname{Pf}(\ulcorner{\delta}\urcorner)$. 
        But then, by Theorem \ref{thm:prf_iff_prf_Pf_code}, $\vdash \delta$,
        which, by soundness, yields $\mathfrak{S} \vDash \delta$.
        This contradicts the assumption that $\mathfrak{S} \not\vDash \delta$.
    \end{itemize}

\end{proof}