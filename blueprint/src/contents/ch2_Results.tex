\chapter{Results of the Theory}

This chapter presents several results from the theory of HF, 
including both theorems of HF and definitions. The latter serve as abbreviations, 
functioning as definitional extensions to the language of HF.

Following \cite{swierczkowski2003finite}, the theorems of HF are presented with informal proofs 
for the sake of efficiency and clarity. 
In Lean, however, formalisation, as the term suggests, requires proofs to be fully formal. 
Fortunately, as explained in the previous chapter, by the completeness theorem 
(Theorem \ref{thm:completeness}), to demonstrate the existence of a formal syntactic proof of 
an $\mathcal{L}$-formula $\phi$, 
it suffices to verify semantically 
(i.e., using only Lean's native logic and not the proof predicate) that 
every type satisfying the axioms of HF also satisfies $\phi$. 
In the latter case, the axioms of HF and $\phi$ are expressed as a proposition type 
({\ttfamily \small Prop}).


More precisely, this mechanism works as follows.
Within the $\mathcal{L}$-structure type class, the non-logical symbols of $\mathcal{L}$ are 
declared for the elements of the class. 
This allows $\mathcal{L}$-formulas to be expressed
as proposition types where each term in the formula becomes an element of an 
arbitrary $\mathcal{L}$-structure. Next, we introduce a new type class, {\ttfamily \small HFSet}, 
which is inhabited by the axioms of HF expressed as proposition types. 
It is then shown that every element of the model-of-HF type class 
(referred to as model type) is an instance of the {\ttfamily \small HFSet} type class.
Consequently, any proposition type that is semantically proven true for every element of 
the {\ttfamily \small HFSet} type class is also true for every model type.
But a proposition type is true for every model type if and only if 
every model of HF is a model of the $\mathcal{L}$-formula corresponding to that proposition type.
Thus, by the completeness theorem, an $\mathcal{L}$-formula is a theorem of HF if its corresponding 
proposition type is true for every element of the {\ttfamily \small HFSet} type class.

We start by showing that most of the ZFC axioms are theorems of HF.

\section{Basic results}

Recall the axioms of HF:

\begin{axiom}
    \label{ax:empty_axiom}
    \lean{HF.Axiom1, HFSet.empty}
    \uses{def:Lang, def:Axioms}
    \leanok
    $z=\varnothing \leftrightarrow \forall x(x \notin z)$.
\end {axiom}

\begin{axiom}
    \label{ax:enlarge_axiom}
    \lean{HF.Axiom2, HFSet.enlarge}
    \uses{def:Lang, def:Axioms}
    \leanok
    $z=x \lhd y \leftrightarrow \forall u(u \in z \leftrightarrow u \in x \lor u=y)$.
\end {axiom}

\begin{axiom}
    \label{ax:induction_axiom}
    \lean{HF.Axiom3, HFSet.induction}
    \uses{def:Lang, def:Axioms}
    \leanok
    $(\alpha(\varnothing) \land \forall x \forall y[\alpha(x) \land \alpha(y) \rightarrow 
    \alpha(x \lhd y)]) \rightarrow \forall x \alpha(x)$. 
\end {axiom}

The assumption in Axiom \ref{ax:induction_axiom} is that 
$\alpha$ is any $\mathcal{L}$-formula which contains a 
distinguished freely occurring variable $z$ such that $x$ and $y$ are substitutable for $z$.

\begin{lemma}
    \label{lem:notin_empty+mem_enlarge+mem_enlarge_empty}
    \lean{HFSet.notin_empty, HFSet.mem_enlarge, HFSet.mem_enlarge_empty}
    \leanok
    \uses{thm:completeness, ax:empty_axiom, ax:enlarge_axiom}
    \leavevmode
    \begin{enumerate}
        \item $x\notin \varnothing$.
        \item $u \in x \lhd y \leftrightarrow u \in x \lor u=y$.
        \item $z\in \varnothing \lhd y \leftrightarrow z = y$.
    \end{enumerate}
\end{lemma}

\begin{proof}
    \leanok
    \leavevmode
    \begin{enumerate}
        \item Substitute $\varnothing$ in Axiom \ref{ax:empty_axiom}.
        \item Substitute $x \lhd y$ in Axiom \ref{ax:enlarge_axiom}.
        \item Substitute $\varnothing$ for $x$ in Lemma 
        \ref{lem:notin_empty+mem_enlarge+mem_enlarge_empty}.2 and use Lemma 
        \ref{lem:notin_empty+mem_enlarge+mem_enlarge_empty}.1.
    \end{enumerate}
\end{proof}

\begin{theorem}[Extensionality Property]
    \label{thm:exten_prop}
    \lean{HFSet.exten_prop}
    \leanok
    \uses{ax:empty_axiom, ax:enlarge_axiom, ax:induction_axiom, 
        lem:notin_empty+mem_enlarge+mem_enlarge_empty}
    $x=z \leftrightarrow \forall u (u \in x \leftrightarrow u \in z)$.
\end{theorem}

\begin{proof}
    \leanok
    Let $\alpha (x)$ be the $\mathcal{L}$-formula
    $x=z \leftrightarrow \forall u (u \in x \leftrightarrow u \in z)$.
    We apply Axiom \ref{ax:induction_axiom}.
    \begin{itemize}
        \item The base case — $\alpha (\varnothing)$ — follows directly from Lemma 
        \ref{lem:notin_empty+mem_enlarge+mem_enlarge_empty}.1 and Axiom \ref{ax:empty_axiom}.
        \item The induction step — 
        $\forall y[\alpha(x) \land \alpha(y) \rightarrow \alpha(x \lhd y)]$ — 
        follows directly from Lemma \ref{lem:notin_empty+mem_enlarge+mem_enlarge_empty}.2 and 
        Axiom \ref{ax:enlarge_axiom}.
    \end{itemize}
\end{proof}

\begin{definition}
    \label{def:singleton+pair}
    \lean{HFSet.insert, HFSet.singleton, HFSet.pair}
    \leanok
    Abbreviate by $\{x\}$, $\{x,y\}$ and $\langle x,y\rangle$ the following terms:
    \begin{enumerate}
        \item $\{x\} = \varnothing \lhd x$.
        \item $\{x,y\} = \{x\} \lhd y$.
        \item $\langle x,y\rangle = \{\{x\}, \{x,y\}\}$.
    \end{enumerate}
\end{definition}

\begin{lemma}
    \label{lem:mem_singleton+...+pair_inj}
    \lean{HFSet.mem_singleton, HFSet.mem_pair, HFSet.duplic_pair_eq_single, HFSet.pair_inj}
    \leanok
    \uses{lem:notin_empty+mem_enlarge+mem_enlarge_empty, thm:exten_prop, def:singleton+pair}
    \leavevmode
    \begin{enumerate}
        \item $ u\in \{x\} \leftrightarrow u = x$.
        \item $u \in \{x,y\}\leftrightarrow u=x \lor u=y$.
        \item $\{x,x\} = \{x\}$.
        \item $\langle x,y\rangle = \langle u,v\rangle \leftrightarrow x=u \land y=v$.
    \end{enumerate}
\end{lemma}

\begin{proof}
    \leanok
    \leavevmode
    \begin{enumerate}
        \item This follows directly from Lemma 
        \ref{lem:notin_empty+mem_enlarge+mem_enlarge_empty}.3.
        \item This follows directly from 
        Lemma \ref{lem:notin_empty+mem_enlarge+mem_enlarge_empty}.2 and 
        Lemma \ref{lem:mem_singleton+...+pair_inj}.1.
        \item This follows directly from Theorem \ref{thm:exten_prop},
        Lemma \ref{lem:mem_singleton+...+pair_inj}.1 and 
        Lemma \ref{lem:mem_singleton+...+pair_inj}.2.
        \item  The proof closely follows an analogous proof from ZF by \cite{enderton1977elements}.
        The $\leftarrow$ direction is trivial.
        For the $\rightarrow$ direction, by the definition of ordered pairs, 
        the hypothesis can be rewritten to
        \begin{equation*}
            \{\{x\}, \{x,y\}\} = \{\{u\}, \{u,v\}\}.
        \end{equation*}
        It follows that
        \begin{equation}
            \{x\} \in \{\{u\}, \{u,v\}\}\quad\text{and}\quad \{x,y\} \in \{\{u\}, \{u,v\}\}
        \end{equation}
        and similarly
        \begin{equation}
            \{u\} \in \{\{x\}, \{x,y\}\}\quad\text{ and }\quad \{u,v\} \in \{\{x\}, \{x,y\}\}.
        \end{equation}
        By Lemma \ref{lem:mem_singleton+...+pair_inj}.2, (2) can be rewritten to
        \begin{equation*}
            (\textbf{a})\quad\{u\}=\{x\}\quad\text{or}\quad (\textbf{b})\quad\{u\}=\{x,y\}
        \end{equation*}
        and
        \begin{equation*}
            (\textbf{c})\quad\{u,v\}=\{x\}\quad\text{or}\quad (\textbf{d})\quad\{u,v\}=\{x,y\}.
        \end{equation*}
        Note that, by Theorem \ref{thm:exten_prop}, Lemma \ref{lem:mem_singleton+...+pair_inj}.1 
        and Lemma \ref{lem:mem_singleton+...+pair_inj}.2,
        \begin{equation}
        \forall x \forall y \forall z (\{x,y\} = \{z\} \leftrightarrow x = y =z).
        \end{equation}
        Now, consider three cases:
        \begin{itemize}
            \item Suppose (\textbf{b}) holds. Then, by (3), $u=x=y$. 
            By Lemma \ref{lem:mem_singleton+...+pair_inj}.3, it then follows that 
            (\textbf{d}) can be rewritten to $\{u,v\}=\{x\}$, 
            implying (\textbf{c}) and (\textbf{d}) are equivalent.
            Thus, by (3), under (\textbf{b}), it must hold that $u=v=x=y$.
            \item Suppose (\textbf{c}) holds. 
            Then, by (3), $u=v=x$. 
            Note that, by (1) and Lemma \ref{lem:mem_singleton+...+pair_inj}.2, 
            either $\{x,y\}=\{u\}$ or $\{x,y\}=\{u,v\}$.  
            By Lemma \ref{lem:mem_singleton+...+pair_inj}.3, it follows that $\{u,v\}=\{u\}$ and 
            thus these two statements are equivalent. 
            Thus, by (3), under (\textbf{c}), it must hold that $x=y=u=v$.
            \item Suppose (\textbf{a}) and (\textbf{d}) hold. From (\textbf{a}) we have $u=x$. 
            From (\textbf{d}) it follows that, by 
            Lemma \ref{lem:mem_singleton+...+pair_inj}.2, $v=x$ or $v=y$. 
            In the latter case, $u=x\land v=y$. 
            If $v=x$, then $u=v=x$, and case (\textbf{c}) holds by (3); 
            this case has already been considered.
        \end{itemize}
    \end{enumerate}
\end{proof}

\begin{theorem}[Existence of the union of two sets]
    \label{thm:exists_union}
    \lean{HFSet.exists_union}
    \leanok
    \uses{ax:induction_axiom, lem:notin_empty+mem_enlarge+mem_enlarge_empty}
    $\exists z \forall u (u \in z \leftrightarrow u \in x \lor u \in y)$.
\end{theorem}

\begin{proof}
    \leanok
    Let $\alpha (x)$ be the $\mathcal{L}$-formula 
    $\exists z \forall u (u \in z \leftrightarrow u \in x \lor u \in y)$.
    We apply Axiom \ref{ax:induction_axiom}.
    \begin{itemize}
        \item For the base case — $\alpha (\varnothing)$ — take $z=y$ and use 
        Lemma \ref{lem:notin_empty+mem_enlarge+mem_enlarge_empty}.1.
        \item For the induction step 
        — $\forall w[\alpha(x) \land \alpha(w) \rightarrow \alpha(x \lhd w)]$ — 
        pick $w$ arbitrarily and assume $\alpha(x)$, i.e. the existence of $x \cup y$. 
        By Lemma \ref{lem:notin_empty+mem_enlarge+mem_enlarge_empty}.2, 
        $\alpha(x \lhd w)$ is equivalent to 
        $\exists z \forall u (u \in z \leftrightarrow u \in x \lor u = w \lor u \in y)$.
        Then, take $z= (x \cup y) \lhd w$ and use 
        Lemma \ref{lem:notin_empty+mem_enlarge+mem_enlarge_empty}.2.
    \end{itemize}
\end{proof}

\begin{theorem}[Existence of the union of a set of sets]
    \label{thm:exists_sUnion}
    \lean{HFSet.exists_sUnion}
    \leanok
    \uses{ax:induction_axiom, lem:notin_empty+mem_enlarge+mem_enlarge_empty, thm:exists_union}
    $\exists z \forall u (u \in z \leftrightarrow \exists (y\in x)[u \in y])$.
\end{theorem}

\begin{proof}
    \leanok
    Let $\alpha (x)$ be the $\mathcal{L}$-formula 
    $\exists z \forall u (u \in z \leftrightarrow \exists (y\in x)[u \in y])$.
    We apply Axiom \ref{ax:induction_axiom}.
    \begin{itemize}
        \item For the base case — $\alpha (\varnothing)$ — take $z=\varnothing$ and use 
        Lemma \ref{lem:notin_empty+mem_enlarge+mem_enlarge_empty}.1.
        \item For the induction step 
        — $\forall w[\alpha(x) \land \alpha(w) \rightarrow \alpha(x \lhd w)]$ — 
        pick $w$ arbitrarily and assume $\alpha(x)$, i.e. the existence of $\bigcup x$.
        We need to find $z$ such that
        \begin{equation*}
        \begin{split}
            u \in z & \leftrightarrow \exists (y \in x \lhd w)[u \in y]\\
            & \leftrightarrow \exists y[(y \in x \lor y = w) \land u \in y]\\
            & \leftrightarrow \exists(y \in x)[u \in y] \lor u \in w\\
            & \leftrightarrow u \in \bigcup x \lor u \in w,
        \end{split}
        \end{equation*}
        using Lemma \ref{lem:notin_empty+mem_enlarge+mem_enlarge_empty}.2. 
        Thus, take $z = (\bigcup x) \cup w$.
    \end{itemize}
\end{proof}

\begin{theorem}[Comprehension Scheme]
    \label{thm:comp_scheme}
    \lean{HFSet.comp_scheme}
    \leanok
    \uses{ax:induction_axiom, lem:notin_empty+mem_enlarge+mem_enlarge_empty}
    $\exists z \forall u (u\in z \leftrightarrow (u \in x) \land \phi (u))$, 
    for any $\mathcal{L}$-formula $\phi$ in which $z$ is not free.
\end{theorem}

\begin{proof}
    \leanok
    Let $\alpha (x)$ be the $\mathcal{L}$-formula 
    $\exists z \forall u (u\in z \leftrightarrow (u \in x) \land \phi (u))$.
    We apply Axiom \ref{ax:induction_axiom}.
    \begin{itemize}
        \item For the base case — $\alpha (\varnothing)$ — take $z=\varnothing$ and use L
        emma \ref{lem:notin_empty+mem_enlarge+mem_enlarge_empty}.1.
        \item For the induction step 
        — $\forall y[\alpha(x) \land \alpha(y) \rightarrow \alpha(x \lhd y)]$ — 
        pick $y$ arbitrarily and assume $\alpha(x)$, i.e. 
        the existence of $ \{u \in x : \phi (u)\}$. We need to find $z$ such that
    \begin{equation*}
    \begin{split}
        u \in z & \leftrightarrow (u \in x \lhd y) \land \phi (u)\\
        & \leftrightarrow (u \in x \lor u = y) \land \phi (u),
    \end{split}
    \end{equation*}
    using Lemma \ref{lem:notin_empty+mem_enlarge+mem_enlarge_empty}.2.
    Since $z$ is not free in $\phi$, take $z = \{u \in x : \phi (u)\} \lhd y$ if $\phi(y)$ 
    and $z=\{u \in x : \phi (u)\}$ if $\neg \phi(y)$.
    \end{itemize}
\end{proof}

\begin{definition}[Intersection]
    \label{def:inter+sInter}
    \lean{HFSet.inter, HFSet.sInter}
    \leanok
    \uses{thm:exists_sUnion, thm:comp_scheme}
    Introduce the notation:
    \begin{enumerate}
        \item $x \cap y = \{u \in x : u \in y\}$.
        \item $\bigcap x = \{u \in \bigcup x : \forall (v \in x)[u \in v] \}$.
    \end{enumerate}  
\end{definition}

\begin{theorem}[Replacement Scheme]
    \label{thm:repl_scheme}
    \lean{HFSet.repl_scheme}
    %\leanok
    \uses{ax:induction_axiom, lem:notin_empty+mem_enlarge+mem_enlarge_empty}
    $$[\forall (u \in x) [\exists! v] \psi (u,v)]\rightarrow 
    \exists z \forall v [v\in z \leftrightarrow \exists(u \in x) \psi (u,v)],$$ 
    for any $\mathcal{L}$-formula $\psi$ in which $z$ is not free.
\end{theorem}

\begin{proof}
    \leanok
    Let $\alpha (x)$ be the $\mathcal{L}$-formula 
    $[\forall (u \in x) [\exists! v] \psi (u,v)]\rightarrow 
    \exists z \forall v [v\in z \leftrightarrow \exists(u \in x) \psi (u,v)]$.
    We apply Axiom \ref{ax:induction_axiom}.
    \begin{itemize}
        \item For the base case — $\alpha (\varnothing)$ — take $z=\varnothing$ and use 
        Lemma \ref{lem:notin_empty+mem_enlarge+mem_enlarge_empty}.1.
        \item For the induction step 
        — $\forall y[\alpha(x) \land \alpha(y) \rightarrow \alpha(x \lhd y)]$ — 
        pick $y$ arbitrarily and assume $\alpha(x)$, i.e. the existence of 
        $\{v : \exists (u\in x)\psi(u,v)\}$ provided that 
        $\forall (u \in x) [\exists! v] \psi (u,v)$.
        Furthermore, assume $\forall (u \in x \lhd y) [\exists! v] \psi (u,v)$, 
        i.e. that $\forall u [(u \in x \lor u = y) \rightarrow \exists! v \psi (u,v)]$.
        Hence, $\exists! v \psi (y,v)$; let $v_y$ be this unique $v$. 
        Moreover, it follows that $\forall (u \in x) [\exists! v] \psi (u,v)$ and 
        thus $\{v : \exists (u\in x)\psi(u,v)\}$ exists.
        By Lemma \ref{lem:notin_empty+mem_enlarge+mem_enlarge_empty}.2, 
        the required $z$ is $\{v : \exists (u\in x)\psi(u,v)\} \lhd v_y$.
    \end{itemize}
\end{proof}

\begin{definition}[Subset relation]
    \label{def:Subset+SSubset}
    \lean{HFSet.Subset, HFSet.SSubset}
    \leanok
    Introduce the abbreviations:
    \begin{enumerate}
        \item $y \subseteq x \leftrightarrow \forall v (v\in y \rightarrow v \in x)$.
        \item $y \subset x \leftrightarrow y \subseteq x \land y \neq x$.
    \end{enumerate}  
\end{definition}

\begin{theorem}[Existence of the power set]
    \label{thm:exists_powerset}
    \lean{HFSet.exists_powerset}
    \leanok
    \uses{ax:induction_axiom, lem:notin_empty+mem_enlarge+mem_enlarge_empty, thm:exten_prop, 
    def:singleton+pair, lem:mem_singleton+...+pair_inj, thm:exists_union, 
    thm:repl_scheme, def:Subset+SSubset}
    $\exists z \forall u (u\in z \leftrightarrow u \subseteq x).$
\end{theorem}

\begin{proof}
    \leanok
    Let $\alpha (x)$ be the $\mathcal{L}$-formula 
    $\exists z \forall u (u\in z \leftrightarrow u \subseteq x)$.
    We apply Axiom \ref{ax:induction_axiom}.
    \begin{itemize}
        \item For the base case — $\alpha (\varnothing)$ — take $z=\{\varnothing\}$ and use 
        Lemma \ref{lem:notin_empty+mem_enlarge+mem_enlarge_empty}, 
        Theorem \ref{thm:exten_prop}, Lemma \ref{lem:mem_singleton+...+pair_inj}.1.
        \item For the induction step 
        — $\forall y[\alpha(x) \land \alpha(y) \rightarrow \alpha(x \lhd y)]$ — 
        pick $y$ arbitrarily and assume $\alpha(x)$, i.e. the existence of the power set $P(x)$.
        One checks directly from the definitions that 
        $$u \subseteq x \lhd y \leftrightarrow u \in P(x) \lor \exists (v \in P(x))[u=v\lhd y].$$
        Indeed, for the $\rightarrow$ direction, the cases $y \notin u$ and $y \in u$ correspond to 
        the left and right cases of the disjunction, respectively, where for the right case, 
        the required $v$ is $u \cap x$. The $\leftarrow$ direction is rather trivial.
        Now, note that the $\mathcal{L}$-formula $\psi(v,u): u = v \lhd y$ satisfies the condition 
        for Theorem \ref{thm:repl_scheme}, i.e. $v \mapsto v \lhd y$ is a well-defined mapping.
        By Theorem \ref{thm:exists_union}, 
        it follows that the required $z$ is $P(x) \cup \{u : \exists (v \in P(x)) [u=v\lhd y]\}$.
    \end{itemize}
\end{proof}

\begin{theorem}[Foundation Property]
    \label{thm:found_prop}
    \lean{HFSet.found_prop}
    \leanok
    \uses{ax:induction_axiom, lem:notin_empty+mem_enlarge+mem_enlarge_empty, thm:exten_prop, 
    def:inter+sInter}
    $z\neq \varnothing \rightarrow \exists (w \in z)[w\cap z = \varnothing]$.
\end{theorem}

\begin{proof}
    \leanok
    We need to show that $\forall (w \in z)[w \cap z \neq \varnothing] \rightarrow z = \varnothing$, 
    which follows if $\forall (w \in z)[w \cap z \neq \varnothing] \rightarrow \forall x \alpha(x)$, 
    where $\alpha(x)$ is the $\mathcal{L}$-formula $(x \notin z) \land (x \cap z = \varnothing)$.
    Thus, assume $\forall (w \in z)[w \cap z \neq \varnothing]$ and pick $x$ arbitrarily.
    We apply Axiom \ref{ax:induction_axiom}.
    \begin{itemize}
        \item For the base case — $\alpha (\varnothing)$ — the assumption immediately implies that 
        $\varnothing \notin z$, as otherwise $\varnothing \cap z \neq \varnothing$.
        \item For the induction step 
        — $\forall y[\alpha(x) \land \alpha(y) \rightarrow \alpha(x \lhd y)]$ — 
        pick $y$ arbitrarily and assume $\alpha(x)$ and $\alpha(y)$.
        Suppose, for a contradiction, that $\neg \alpha (x \lhd y)$, i.e. 
        that $(x \lhd y) \cap z \neq \varnothing \lor (x \lhd y) \in z$. 
        As $y \notin z$ by $\alpha(y)$, the first case implies $x \cap z \neq \varnothing$, 
        which contradicts $\alpha(x)$. 
        By the assumption, the second case implies $(x \lhd y) \cap z \neq \varnothing$, 
        which is the first case.
    \end{itemize}
\end{proof}

\begin{corollary}
    \label{cor:mem_irrefl}
    \lean{HFSet.mem_irrefl}
    \leanok
    \uses{lem:notin_empty+mem_enlarge+mem_enlarge_empty, thm:exten_prop, 
    def:singleton+pair, lem:mem_singleton+...+pair_inj, def:inter+sInter, thm:found_prop}
    $x \notin x$.
\end{corollary}

\begin{proof}
    \leanok
    Apply Theorem \ref{thm:found_prop} to $z = \{x\}\neq \varnothing$.
    It follows that $\exists (w \in \{x\})[w\cap \{x\} = \varnothing]$, 
    which implies $x \cap \{x\} = \varnothing$.
\end{proof}

\section{Ordinals}

\begin{definition}
    \label{def:IsTrans+IsOrd}
    \lean{HFSet.IsTrans, HFSet.IsOrd, HFSet.Ord}
    \leanok
    \uses{def:Subset+SSubset}
    We say that $x$ is \textit{transitive} if every element of $x$ is a subset of $x$. 
    We call $x$ an \textit{ordinal} if $x$ is transitive and every element of $x$ is transitive, 
    abbreviated by $\operatorname{Ord}(x)$.

    Ordinals will be mostly denoted by $k$, $l$, $m$, $n$, $p$, $q$, $r$.
\end{definition}

\begin{definition}[Successor]
    \label{def:succ}
    \lean{HFSet.succ, HFSet.Ord.succ}
    \leanok
    The \textit{successor} of $x$ is defined as $x \lhd x$, abbreviated by $\operatorname{succ}(x)$.
\end{definition}

\begin{lemma}
    \label{lem:IsOrd.succ_isOrd+...+IsOrd.empty_is_mem}
    \lean{HFSet.IsOrd.succ_isOrd, HFSet.IsOrd.mem_isOrd, HFSet.empty_isOrd, 
    HFSet.IsOrd.empty_of_mem_and_disjoint, HFSet.IsOrd.empty_is_mem, HFSet.Ord.empty_is_lt}
    \leanok
    \uses{ax:empty_axiom, lem:notin_empty+mem_enlarge+mem_enlarge_empty, 
    thm:exten_prop, def:inter+sInter, thm:found_prop, def:IsTrans+IsOrd, def:succ}
    \leavevmode
    \begin{enumerate}
        \item $\operatorname{Ord}(k) \rightarrow \operatorname{Ord}(\operatorname{succ}(k))$.
        \item $\operatorname{Ord}(k) \land l \in k \rightarrow \operatorname{Ord}(l)$.
        \item $\operatorname{Ord}(\varnothing)$.
        \item $(\operatorname{Ord}(k) \land (k \neq \varnothing) \land 
        (l \cap k = \varnothing))\rightarrow(l \in k \leftrightarrow l = \varnothing)$.
    \end{enumerate}
\end{lemma}

\begin{proof}
    \leanok
    \leavevmode
    The proofs of 1, 2 and 3 are immediate from Definitions \ref{def:IsTrans+IsOrd} and 
    \ref{def:succ}.
    For 4, assume $k$ is a non-empty ordinal, $l$ is such that $l \cap k = \varnothing$, 
    and suppose $l \in k$. As $k$ is an ordinal, $l \subseteq k$, and as $l \cap k = \varnothing$, 
    it follows that $l=\varnothing$.
    Conversely, as $k \neq \varnothing$, by Theorem \ref{thm:found_prop} there must exist such 
    an $l \in k$ satisfying $l \cap k = \varnothing$. Hence, $\varnothing \in k$.
\end{proof}

\begin{theorem}[Comparability of ordinals]
    \label{thm:IsOrd.comparability}
    \lean{HFSet.IsOrd.comparability, HFSet.Ord.comparability}
    \leanok
    \uses{ax:empty_axiom, lem:notin_empty+mem_enlarge+mem_enlarge_empty, thm:comp_scheme, 
    def:inter+sInter, def:Subset+SSubset, thm:exists_powerset, thm:found_prop, 
    cor:mem_irrefl, def:IsTrans+IsOrd, lem:IsOrd.succ_isOrd+...+IsOrd.empty_is_mem}
    Let $k,l$ be ordinals. Then, $k \in l \lor k = l \lor l \in k$.
\end{theorem}

\setcounter{equation}{0}
\begin{proof}
    \leanok
    Denote the $\mathcal{L}$-formula $k \in l \lor k = l \lor l \in k$ by $\beta (k,l)$. 
    Suppose, for a contradiction, there exist ordinals $k$ and $l$ such that $\neg \beta (k,l)$, 
    i.e. $k \notin l \land k \neq l \land l \notin k$. 
    We claim there exists an ordinal $k_0$ that satisfies
    \begin{equation}\label{eq:1}
    \forall(m \in k_0)[\forall l \beta(m,l)] \land \exists l \neg \beta(k_0,l),
    \end{equation}
    where $\forall$ and $\exists$ range over ordinals. 
    Indeed, if there exist ordinals $k$ and $l$ such that $\neg \beta (k,l)$, then 
    $$K := \{m \subseteq k : \operatorname{Ord}(m) \land \exists l (\operatorname{Ord}(l) \land 
    \neg \beta (m,l))\} \neq \varnothing.$$
    By Theorem \ref{thm:found_prop}, there exists $k_0 \in K$ such that $k_0 \cap K = \varnothing$. 
    We claim this $k_0$ satisfies (\ref{eq:1}). 
    Clearly, $\exists l \neg \beta(k_0,l)$. 
    Now suppose, for a contradiction, there exist ordinals $r_0 \in k_0$ and $l_0$ such that 
    $\neg \beta (r_0,l_0)$. 
    Then, as $k_0 \subseteq k$, $r_0 \in k$, and thus $r_0 \subseteq k$. 
    Moreover, $r_0 \in K$, which contradicts $k_0 \cap K = \varnothing$. 
    
    We claim further that, having selected $k_0$ that satisfies (\ref{eq:1}), 
    there exists an ordinal $l_0$ that satisfies 
    \begin{equation}\label{eq:2}
    \forall(p \in l_0)\beta(k_0,p) \land \neg \beta(k_0,l_0),
    \end{equation}
    where $\forall$ ranges over ordinals. 
    Indeed, if there exists an ordinal $l$ such that $\neg \beta (k_0,l)$, then 
    $$L := \{p \subseteq l : \operatorname{Ord}(p) \land \neg \beta (k_0, p)\} \neq \varnothing.$$
    By Theorem \ref{thm:found_prop}, there exists $l_0 \in L$ such that $l_0 \cap L = \varnothing$. 
    We claim this $l_0$ satisfies (\ref{eq:2}). Clearly, $\neg \beta(k_0,l_0)$. 
    Now suppose, for a contradiction, there exists an ordinal $q_0 \in l_0$ such that 
    $\neg \beta (k_0, q_0)$. 
    Then, as $l_0 \subseteq l$, $q_0 \in l$, and thus $q_0 \subseteq l$. 
    Moreover, $q_0 \in L$, which contradicts $l_0 \cap L = \varnothing$. 
    
    We now claim that $l_0 \subset k_0$. Indeed, suppose $p \in l_0$. 
    Then $\beta (k_0, p)$, by (\ref{eq:2}), i.e. $k_0 \in p \lor k_0 = p \lor p \in k_0$. 
    But both $k_0 \in p$ and $k_0 = p$ contradict $\neg \beta(k_0,l_0)$ in (\ref{eq:2}), 
    as $p \in l_0$ is assumed. Thus $p \in k_0$, and $l_0 \subseteq k_0$ follows. 
    Since $l_0 = k_0$ also contradicts $\neg \beta(k_0,l_0)$ in (\ref{eq:2}), 
    we get $l_0 \subset k_0$. \\
    
    Let $m \in k_0 \setminus l_0$. Then $\beta(m, l_0)$, by (\ref{eq:1}), i.e.
    $$m \in l_0 \lor m = l_0 \lor l_0 \in m.$$
    It follows immediately that $m \in l_0$ is impossible. 
    But both $m=l_0$ and $l_0 \in m$ contradict $\neg \beta(k_0,l_0)$ in (\ref{eq:2}), 
    as $m \in k_0$.
\end{proof}

\begin{corollary}
    \label{cor:IsOrd.exclusive_comparability+...+IsOrd.succ_inj}
    \lean{HFSet.IsOrd.exclusive_comparability, HFSet.Ord.exclusive_comparability, 
    HFSet.IsOrd.sSubset_iff_mem, HFSet.Ord.sSubset_iff_lt, HFSet.IsOrd.succ_eq_or_succ_mem, 
    HFSet.Ord.succ_eq_or_succ_lt, HFSet.IsOrd.succ_inj, HFSet.Ord.succ_inj}
    \leanok
    \uses{lem:notin_empty+mem_enlarge+mem_enlarge_empty, thm:exten_prop, def:Subset+SSubset, 
    cor:mem_irrefl, def:IsTrans+IsOrd, def:succ, lem:IsOrd.succ_isOrd+...+IsOrd.empty_is_mem, 
    thm:IsOrd.comparability}
    Let $k,l$ be ordinals. Then,
    \begin{enumerate}
        \item Exactly one of $k\in l$, $k=l$, $l \in k$ occurs.
        \item $k\in l \leftrightarrow k \subset l$.
        \item $l \in k \rightarrow (\operatorname{succ}(l)=k \lor \operatorname{succ}(l)\in k)$.
        \item $(\operatorname{succ}(k) = \operatorname{succ}(l) )\rightarrow k = l$.
    \end{enumerate}
\end{corollary}

\begin{proof}
    \leanok
    \leavevmode
    \begin{enumerate}
        \item Any two of the three mentioned possibilities contradict 
        Corollary \ref{cor:mem_irrefl}. 
        At least one possibility occurs by Theorem \ref{thm:IsOrd.comparability}.
        \item The $\rightarrow$ direction follows by the transitivity of $l$ and 
        Corollary \ref{cor:mem_irrefl}. 
        Suppose $k \subset l$. Then $k \in l \lor l \in k$ by Theorem \ref{thm:IsOrd.comparability}. 
        But $l \in k$ yields $l \in l$, contradicting Corollary \ref{cor:mem_irrefl}.
        \item By Theorem \ref{thm:IsOrd.comparability}, we have to exclude $k \in 
        \operatorname{succ}(l)$, i.e. we have to show that neither $k\in l$ nor $k=l$ can occur. 
        This is indeed so because each of these possibilities, together with $l\in k$, 
        leads to $l \in l$.
        \item Suppose, for a contradiction, that $k \neq l$. 
        Then $\operatorname{succ}(k) = k \lhd k = l \lhd l = \operatorname{succ}(l)$ 
        implies $k \in l$ and $l \in k$, yielding again $l \in l$.
    \end{enumerate}
\end{proof}

\begin{definition}
    \label{def:Ord.lt+Ord.le}
    \lean{HFSet.Ord.lt, HFSet.Ord.le}
    \leanok
    For ordinals $k,l$, we shall use:
    \begin{enumerate}
    \item $k\,<\,l$ to abbreviate $k \in l$,
    \item $k \leq l$ to abbreviate $k\,<\,l \lor k = l$.
    \end{enumerate}
\end{definition}

\begin{theorem}
    \label{thm:Ord.le_totalOrder+Ord.lt_sTotalOrder}
    \lean{HFSet.Ord.le_totalOrder, HFSet.Ord.lt_sTotalOrder}
    \uses{cor:mem_irrefl, def:IsTrans+IsOrd, thm:IsOrd.comparability, def:Ord.lt+Ord.le}
    \leanok
    On every set of ordinals, the relation $\leq$ ($<$) is a (strict) total order.
\end{theorem}

\begin{proof}
    \leanok
    Reflexivity of $\leq$ follows directly from $=$ being reflexive; 
    irreflexivity of $<$ follows directly from Corollary \ref{cor:mem_irrefl}.
    Transitivity is immediate from Definition \ref{def:IsTrans+IsOrd}.
    Antisymmetry and asymmetry are straightforward from Definition \ref{def:IsTrans+IsOrd} and
    Corollary \ref{cor:mem_irrefl}.
    (Strong) connectivity is derived trivially from Theorem \ref{thm:IsOrd.comparability}.
\end{proof}

\begin{theorem}
    \label{thm:IsOrd.exists_max_of_set+IsOrd.exists_min_of_set}
    \lean{HFSet.IsOrd.exists_max_of_set, HFSet.IsOrd.exists_min_of_set}
    \leanok
    \uses{ax:induction_axiom, lem:notin_empty+mem_enlarge+mem_enlarge_empty, def:IsTrans+IsOrd, 
    thm:IsOrd.comparability}
    If $x$ is a non-empty set of ordinals, then $x$ has a smallest and a largest element, 
    i.e.
    \begin{equation*}
    \exists(l \in x) \forall (k \in x) [l \leq k]\quad\text{and}\quad\exists(l \in x) 
    \forall (k \in x) [k \leq l].
    \end{equation*}
\end{theorem}

\begin{proof}
    \leanok
    We first prove the existence of the largest element of $x$, denoted by $\max(x)$. 
    Let $\alpha(x)$ be the $\mathcal{L}$-formula 
    $$(\forall (k \in x)[\operatorname{Ord} k]) \rightarrow (x \neq \varnothing) \rightarrow 
    (\exists(l \in x) \forall (k \in x) [k \leq l]).$$ 
    We apply Axiom \ref{ax:induction_axiom}.
    \begin{itemize}
        \item For the base case — $\alpha (\varnothing)$ — obviously, $x = \varnothing$ immediately 
        contradicts $x \neq \varnothing$.
        \item For the induction step 
        — $\forall y[\alpha(x) \land \alpha(y) \rightarrow \alpha(x \lhd y)]$ — 
        pick $y$ arbitrarily and assume $\alpha(x)$, 
        i.e. the existence of $\max(x)$ if $x$ is a set of ordinals and $x \neq \varnothing$. 
        Assume $x \lhd y$ is a set of ordinals, evidently non-empty. 
        Then $x$ is a set of ordinals or it is empty.
        If $x=\varnothing$, then $x \lhd y = \{y\}$, and $\max(x\lhd y) = y$.
        Now suppose $x \neq \varnothing$, and thus $\max(x)$ exists. 
        Since $x \lhd y$ arises from $x$ by adjoining one element, it is clear that 
        $\max (x \lhd y)$ also exists. 
        More specifically, it equals $y$ if $\max (x) \in y$, and $\max (x)$ otherwise.
    \end{itemize}
    To deduce the existence of $\min (x)$, 
    replace in the above proof $\leq$ by $\geq$ and "max" by "min" throughout.
\end{proof}

\begin{corollary}
    \label{cor:IsOrd.exists_pred}
    \lean{HFSet.IsOrd.exists_pred}
    \leanok
    \uses{def:IsTrans+IsOrd, def:succ, lem:IsOrd.succ_isOrd+...+IsOrd.empty_is_mem, 
    cor:IsOrd.exclusive_comparability+...+IsOrd.succ_inj, 
    thm:IsOrd.exists_max_of_set+IsOrd.exists_min_of_set}
    $(\operatorname{Ord}(k) \land k \neq \varnothing) \rightarrow 
    \exists!l(\operatorname{succ}(l)=k)$.
\end{corollary}

\begin{proof}
    \leanok
    By Lemma \ref{lem:IsOrd.succ_isOrd+...+IsOrd.empty_is_mem}.2, 
    $k$ is a non-empty set of ordinals. 
    Thus, by Theorem \ref{thm:IsOrd.exists_max_of_set+IsOrd.exists_min_of_set}, 
    $\max (k)$ exists, where $\max (k) \in k$. 
    It follows from Corollary \ref{cor:IsOrd.exclusive_comparability+...+IsOrd.succ_inj}.3 that 
    either $\operatorname{succ}(\max (k)) = k$ or $\operatorname{succ}(\max (k)) \in k$. 
    However, the latter contradicts the definition of $\max (k)$. 
    Thus $\operatorname{succ}(\max (k)) = k$. Uniqueness follows from 
    Corollary \ref{cor:IsOrd.exclusive_comparability+...+IsOrd.succ_inj}.4.
\end{proof}

\begin{definition}[Predecessor]
    \label{def:IsOrd.pred}
    \lean{HFSet.IsOrd.pred, HFSet.Ord.pred}
    \leanok
    \uses{def:succ, cor:IsOrd.exists_pred}
    The unique (if existing) $l$ for which $\operatorname{succ}(l) = k$ is called the 
    \textit{predecessor} of $k$, abbreviated by $\operatorname{pred}(k)$.
\end{definition}

\section{\textit{p}-functions}

\begin{definition}
    \label{def:IsFunc+IsFunc.dom}
    \lean{HFSet.IsFunc, HFSet.IsFunc.dom}
    \leanok
    \uses{def:singleton+pair, thm:exists_sUnion, thm:comp_scheme}
    We say that $x$ is a \textit{function} if every element of $x$ is a an ordered pair and
    $$ 
    \left(\left\langle u, v_1\right\rangle \in x \wedge\left\langle u, v_2\right\rangle 
    \in x\right) \rightarrow v_1=v_2.
    $$
    Note that $\langle u, v\rangle = \{\{u\}, \{u,v\}\} \in x$ implies $\{u\} \in \bigcup x$, and 
    hence $u \in \bigcup\left(\bigcup x\right)$.
    Thus, the \textit{domain} of a function $x$ is defined as
    $$
    \operatorname{dom}(x)=\{u \in \bigcup\left(\bigcup x\right): 
    \exists v(\langle u, v\rangle \in x)\}
    $$
    (see Theorems \ref{thm:exists_sUnion} and \ref{thm:comp_scheme}).
\end{definition}

\begin{definition}[Sequence]
    \label{def:IsSeq}
    \lean{HFSet.IsSeq}
    \leanok
    \uses{def:IsTrans+IsOrd, def:IsFunc+IsFunc.dom}
    We call $s$ a \textit{sequence} if $s$ is a function and $\operatorname{dom}(s)=k$ 
    where $k$ is a non-empty ordinal, abbreviated by $\operatorname{Seq}(s,k)$.

    If $\operatorname{Seq}(s,k)$, then, for any $n\,<\,k$, we write $s_n$ to abbreviate the 
    uniquely existing $z$ such that $\langle n, z\rangle \in s$.
\end{definition}

\begin{definition}[Functional]
    \label{def:IsFunctional}
    \lean{HFSet.IsFunctionalUnary, HFSet.IsFunctionalBinary, HFSet.IsFunctionalTernary}
    \leanok
    We say an $\mathcal{L}$-formula $\phi$ with $n+1$ free variables is \textit{functional} 
    with respect to a freely occurring variable $y$ if 
    $$
    \forall x_1 \ldots \forall x_n \exists! y  \phi (x_1,\ldots, x_n, y).
    $$
\end{definition}

\begin{definition}[\textit{p}-function]
    \label{def:pFunc}
    \lean{HFSet.pFuncUnary, HFSet.pFuncBinary, HFSet.pFuncTernary}
    \leanok
    \uses{def:IsFunctional}
    If $\phi$ has $n+1$ free variables and is functional with respect to $y$, 
    we associate with $\phi$ and $y$ a new $n$-ary function symbol $F_\phi^y$, 
    called a \textit{p-function}. 
    However, formally, this function symbol is not part of the language $\mathcal{L}$. 
    In particular, $F_\phi^y(x_{i_1},\ldots,x_{i_n})=u$ abbreviates
    $$
    \exists w (w = u \land \phi (x_{i_1},\ldots,x_{i_n},w)),
    $$
    where the latter, in view of $\exists! y \phi (x_{i_1},\ldots,x_{i_n},y)$, 
    is equivalent to $\phi (x_{i_1},\ldots,x_{i_n},u)$.

    Examples of \textit{p}-functions are:
    \begin{itemize}
        \item $F_\phi^y(x) = \bigcup x$, with $\phi(x,y)$ defined as $y = \bigcup x$.
        \item $F_\psi^y(x) = \bigcap x$, with $\psi(x,y)$ defined as $y = \bigcap x$.
        \item $F_\chi^y(x) = P (x)$, with $\chi(x,y)$ defined as $y = P (x)$
        ($P(x)$ is the power set of $x$).
    \end{itemize}

    Often, instead of writing $F_\phi^y$, we shall use only one capital letter to 
    denote a \textit{p}-function. 
    In the non-arbitrary case, $\phi$ and $y$ will be specified beforehand.
\end{definition}

\begin{theorem}
    \label{thm:Ord.exists_pFuncRecursive}
    \uses{def:singleton_pair, thm:exists_union, thm:comp_scheme, def:IsTrans+IsOrd, 
    lem:IsOrd.succ_isOrd+...+IsOrd.empty_is_mem, def:Ord.lt+Ord.le, 
    thm:IsOrd.exists_max_of_set+IsOrd.exists_min_of_set, def:IsOrd.pred,
    def:IsFunc+IsFunc.dom, def:IsSeq, def:IsFunctional, def:pFunc}
    Let $k$ be an ordinal. Then, for every constant term $\tau$ and \textit{p}-function $G$,
    there exists a \textit{p}-function $F$ such that:
    \begin{enumerate}
        \item $k=\varnothing \rightarrow F(k) = \tau$.
        \item $k\neq \varnothing \rightarrow F(k) = G(F(\operatorname{pred}(k)))$.
    \end{enumerate}
\end{theorem}

\begin{proof}
    Let $\phi(k,y)$ be the $\mathcal{L}$-formula
    \begin{equation*}
        \begin{split}
    & (k \neq \varnothing \land \exists s [\operatorname{Seq}(s,k) \land 
    y = G(s_{\operatorname{pred}(k)}) \land \forall (n\,<\,k)[(n=\varnothing \land s_n = \tau)
    \lor\\ &(n \neq \varnothing \land s_n = G(s_{\operatorname{pred}(n)}))]])
    \lor (k = \varnothing \land y = \tau).
        \end{split}
    \end{equation*}
    We claim that $\phi$ is functional with respect to $y$.
    Indeed, if $k=\varnothing$, then trivially $y=\tau$ is unique.
    Now suppose, for a contradiction, there exists an ordinal $k\neq \varnothing$ such that there
    is no unique $y$ for which $\phi(k,y)$. 
    Then, by Theorems \ref{thm:comp_scheme} and 
    \ref{thm:IsOrd.exists_max_of_set+IsOrd.exists_min_of_set}, there exists a
    least ordinal $m\neq\varnothing$ such that there is no unique $y$ for which $\phi(m,y)$.
    Let $p=\operatorname{pred}(m)$; it follows there exists a unique $y$ such that $\phi(p,y)$.
    Suppose $p\neq \varnothing$. Then there exists $s$ such that
    \begin{equation*}
        \begin{split}
    &\operatorname{Seq}(s,p) \land y = G(s_{\operatorname{pred}(p)}) \land 
    \forall (n\,<\,p)[(n=\varnothing \land s_n = \tau) \lor \\& 
    (n \neq \varnothing \land s_n = G(s_{\operatorname{pred}(n)}))].
        \end{split}
    \end{equation*}
    But then $\tilde{s} = s \cup \langle p, G(s_{\operatorname{pred}(p)}) \rangle$ satisfies
    \begin{equation*}
    \operatorname{Seq}(\tilde{s},m) \land \forall (n\,<\,m)[(n=\varnothing \land \tilde{s}_n = \tau) 
    \lor (n \neq \varnothing \land \tilde{s}_n = G(\tilde{s}_{\operatorname{pred}(n)}))], 
    \end{equation*}
    and thus $y = G(\tilde{s}_p)$ uniquely satisfies $\phi(m,y)$, a contradiction.
    Now, suppose $p=\varnothing$. Then $\bar{s}=\langle \varnothing, \tau \rangle$ satisfies
    \begin{equation*}
        \operatorname{Seq}(\bar{s},m) \land \forall (n\,<\,m)[(n=\varnothing \land \bar{s}_n = \tau) 
        \lor (n \neq \varnothing \land \bar{s}_n = G(\bar{s}_{\operatorname{pred}(n)}))], 
    \end{equation*}
    and thus $y = G(\bar{s}_\varnothing)$ uniquely satisfies $\phi(m,y)$, again a contradiction.
    
    It is easily verified that \ref{thm:Ord.exists_pFuncRecursive}.1 and 
    \ref{thm:Ord.exists_pFuncRecursive}.2 hold when $F$ is $F_\phi^y$.
\end{proof}

\begin{definition}
    \label{def:Ord.pFuncRecursive}
    \uses{thm:Ord.exists_pFuncRecursive}
    Say the \textit{p}-function $F$ in Theorem \ref{thm:Ord.exists_pFuncRecursive} is
    \textit{defined recursively on ordinals}.
\end{definition}

\section{The rank function}

\begin{definition}
    \label{def:Ord.R}
    \uses{def:Ord.pFuncRecursive}
    Let $R(x)$ denote the $p$-function defined recursively on ordinals by
    $$
    R(x) = \begin{cases}
        \varnothing & \text{if } x = \varnothing, \\
        P(R(\operatorname{pred}(x))) & \text{otherwise}.
    \end{cases}
    $$
\end{definition}

\begin{lemma}
    \label{lem:Ord.R.sSubset_of_lt+Ord.R.isTrans}
    \uses{def:Subset+SSubset, def:Ord.lt+Ord.le, 
        thm:IsOrd.exists_max_of_set+IsOrd.exists_min_of_set, def:Ord.R}
    For all ordinals $m,n$ :
    \begin{enumerate}
        \item $n < m \rightarrow R(n) \subset R(m)$.
        \item $R(m)$ is transitive.
    \end{enumerate}
\end{lemma}

\begin{proof}
    \begin{enumerate}
        \item Clearly $R(\varnothing) = \varnothing \subset R(m)$ for all $m>\varnothing$.
        Now if, for some $m$, there exists some $n<m$ but $R(n) \subseteq R(m)$ fails,
        then by Theorem \ref{thm:IsOrd.exists_max_of_set+IsOrd.exists_min_of_set} there is a
        smallest $n$ with that property.
        As $n>\varnothing$, it follows that $\operatorname{pred}(n)<\operatorname{pred}(m)$, and thus 
        $R(\operatorname{pred}(n)) \subset R(\operatorname{pred}(m))$.
        But, by Definition \ref{def:Ord.R}, this implies $R(n) \subset R(m)$, a contradiction.
        
        \item Clearly $R(\varnothing) = \varnothing$ is transitive.
        Now, if $m\neq \varnothing$, then $\operatorname{pred}(m)<m$ and thus 
        $$
        x \in R(m) \rightarrow x \subseteq R(\operatorname{pred}(m)) \subset R(m),
        $$
        by Definition \ref{def:Ord.R} and Lemma \ref{lem:Ord.R.sSubset_of_lt+Ord.R.isTrans}.1.
    \end{enumerate}
\end{proof}

\begin{theorem}
    \label{thm:forall_exists_ord_mem_R}
    \uses{ax:induction_axiom, def:Ord.R, lem:Ord.R.sSubset_of_lt+Ord.R.isTrans}
    $\exists n [\operatorname{Ord}(n) \land x \in R(n)]$.
\end{theorem}

\begin{proof}
    Let $\alpha (x)$ be the $\mathcal{L}$-formula 
    $\exists n [\operatorname{Ord}(n) \land x \in R(n)]$.
    We apply Axiom \ref{ax:induction_axiom}.
    \begin{itemize}
        \item For the base case — $\alpha (\varnothing)$ — the required ordinal is 
        $\operatorname{succ}(\varnothing)$, as $\varnothing \in P(R(\varnothing)) = P(\varnothing)
        = \{\varnothing\}$.
        \item For the induction step 
        — $\forall y[\alpha(x) \land \alpha(y) \rightarrow \alpha(x \lhd y)]$ — 
        pick $y$ arbitrarily and assume $\alpha(x) \land \alpha(y)$, i.e. $x \in R(n)$, 
        $y \in R(m)$ for some ordinals $m,n$. 
        Then, by Lemma \ref{lem:Ord.R.sSubset_of_lt+Ord.R.isTrans}.2,
        $$
        x \lhd y = x \cup \{y\} \subseteq R(n) \cup R(\operatorname{succ}(m)).
        $$
        So, by Lemma \ref{lem:Ord.R.sSubset_of_lt+Ord.R.isTrans}.1, 
        for $k=\max \{\operatorname{succ}(m), n\}$, $x\lhd y \subseteq R(k)$, and thus
        $x \lhd y \in R(\operatorname{succ}(k))$.
    \end{itemize}
\end{proof}

\begin{definition}[Rank function]
    \label{def:rank}
    \uses{def:Ord.R}
    Let $\operatorname{rank}(x)$ denote the least ordinal $n$ such that 
    $x \in R(\operatorname{succ}(n))$, i.e. $x \subseteq R(n)$.
\end{definition}

\begin{theorem}
    \label{thm:rank.lt_of_mem}
    \uses{def:Ord.R, def:rank}
    $x \in y \rightarrow \operatorname{rank}(x) < \operatorname{rank}(y)$.
\end{theorem}

\begin{proof}
    Let $x \in y$ and suppose $\operatorname{rank}(y)=n$.
    Then $y \subseteq R(n)$, and thus $x \in R(n)$.
    It follows that $n \neq \varnothing$ and $x \subseteq R(\operatorname{pred}(n))$.
    Hence, $\operatorname{rank}(x) \leq \operatorname{pred}(n) < n$.
\end{proof}

\begin{definition}[Transitive closure]
    \label{def:cl}
    \uses{def:inter+sInter,lem:Ord.R.sSubset_of_lt+Ord.R.isTrans, thm:forall_exists_ord_mem_R}
    The \textit{transitive closure} of $x$, denoted by $\operatorname{cl}(x)$, is the minimal 
    transitive set $y$ such that $x \subseteq y$.
\end{definition}

\begin{theorem}
    \label{thm:rank_eq_rank_cl}
    \uses{ax:induction_axiom, lem:Ord.R.sSubset_of_lt+Ord.R.isTrans, thm:rank.lt_of_mem, def:cl}
    $\operatorname{rank}(x) = \operatorname{rank}(\operatorname{cl}(x))$.
\end{theorem}

\begin{proof}
    Under construction.
\end{proof}

\begin{definition}[Restriction]
    \label{def:IsFunc.restr+pFunc.restr}
    \uses{thm:repl_scheme, def:IsFunc+IsFunc.dom}
    Let $f$ be a function and $z \subseteq \operatorname{dom}(f)$.
    Then the function $\{y \in f : \exists (u \in z) \exists v (y = \langle u,v \rangle)\}$
    will be denoted by $f \upharpoonright z$ and called the \textit{restriction} of $f$ to $z$.

    Similarly, if $F$ is a \textit{p}-function, then $F \upharpoonright z$ denotes the 
    restriction of $F$ to $z$.
\end{definition}

\begin{theorem}[Recursion on rank]
    \label{thm:rank.exists_pFuncRecursive}
    \uses{thm:found_prop, def:pFunc, thm:rank.lt_of_mem, thm:rank_eq_rank_cl, def:cl, 
        def:IsFunc.restr+pFunc.restr}
    For every binary $p$-function $G(x,y)$, there exists a unary $p$-function $F$ such that
    $$
    F(x) = G(x, F \upharpoonright \operatorname{cl}(x)).
    $$
\end{theorem}

\begin{proof}
    Under construction.
\end{proof}

