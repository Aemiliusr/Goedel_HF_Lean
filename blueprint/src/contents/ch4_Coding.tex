\chapter{Gödel Coding}

In the general sense, a Gödel coding is a function that assigns to each symbol, each term and
each formula of some formal language a unique code, called its Gödel code.
More text on coding.

\begin{definition}
    \lean{HF.IsInΓ0, HF.IsInΓ, HF.Code}
    \label{def:IsInΓ0+IsInΓ+Code}
    \leanok
    \uses{def:Lang, def:C}
    We denote by $\Gamma_0$ the least family of constant terms such that
    \begin{enumerate}
        \item $\varnothing \in \Gamma_0$.
        \item If $\sigma \in \Gamma_0$, then $\sigma \lhd \sigma \in \Gamma_0$.
    \end{enumerate}
    Furthermore, we denote by $\Gamma$ the least family of constant terms such that 
    $\Gamma_0 \subseteq \Gamma$ and 
    \begin{enumerate}\setcounter{enumi}{2} %does not work in web
        \item If $\sigma, \tau \in \Gamma$, then $\langle\sigma, \tau\rangle \in \Gamma$.
    \end{enumerate}
\end{definition}

\begin{theorem}
    \lean{HF.ne_of_isInΓ_and_distinct}
    \label{thm:ne_of_isInΓ_and_distinct}
    \leanok
    \uses{def:C, def:IsInΓ0+IsInΓ+Code}
    Let $\sigma, \tau \in \Gamma$ be distinct constant terms, i.e. they are different strings of
    symbols. Then $\vdash \sigma \neq \tau$.
\end{theorem}

\begin{proof}
    Under construction.
\end{proof}

Need text on what langle rangle with more than 2 inputs means.

\begin{definition}[Coding of symbols]
    \lean{HF.Code.Mem, HF.Code.Enlarge, HF.Code.Eq, HF.Code.Or, HF.Code.Neg, HF.Code.Ex}
    \label{def:Code.Symbols}
    \leanok
    \uses{def:IsInΓ0+IsInΓ+Code}
    We begin by assigning to each symbol of the language a unique code, denoted by
    $\ulcorner \cdot\urcorner$
    In particular, these are the following constant terms belonging to $\Gamma$:
    $$
    \begin{aligned}
    \ulcorner{\varnothing}\urcorner & =\varnothing, \\
    \ulcorner{\in}\urcorner & =\langle \varnothing,\varnothing\rangle, \\
    \ulcorner{\lhd}\urcorner & =\langle \varnothing,\varnothing,\varnothing\rangle, \\
    \ulcorner{=}\urcorner & =\langle \varnothing,\varnothing,\varnothing,\varnothing\rangle, \\
    \ulcorner{\lor}\urcorner & =
    \langle \varnothing,\varnothing,\varnothing,\varnothing,\varnothing\rangle, \\
    \ulcorner{\neg}\urcorner & =
    \langle \varnothing,\varnothing,\varnothing,\varnothing,\varnothing,\varnothing\rangle, \\
    \ulcorner{\exists}\urcorner & =
    \langle \varnothing,\varnothing,\varnothing,\varnothing,\varnothing,\varnothing,
    \varnothing\rangle,\\
    \ulcorner{x_1}\urcorner & = \varnothing \lhd \varnothing, \quad 
    \ulcorner{x_2}\urcorner = \ulcorner{x_1}\urcorner \lhd \ulcorner{x_1}\urcorner,\quad\ldots\quad,
    \ulcorner{x_{k+1}}\urcorner = \ulcorner{x_k}\urcorner \lhd \ulcorner{x_k}\urcorner.
    \end{aligned}
    $$
\end{definition}

\begin{definition}[Coding of terms]
    \lean{HF.Code.Term}
    \label{def:Code.Term}
    \uses{def:IsInΓ0+IsInΓ+Code, def:Code.Symbols}
    The terms $\varnothing$ and $x_k$ have been code already. 
    The remaining case follows the inductive definition
    $$
    \ulcorner{\sigma \lhd \tau}\urcorner = 
    \langle\ulcorner{\lhd}\urcorner, \ulcorner{\sigma}\urcorner, \ulcorner{\tau}\urcorner \rangle.
    $$
\end{definition}

\begin{definition}[Coding of formulas]
    \lean{HF.Code.Formula}
    \label{def:Code.Formula}
    \leanok
    \uses{def:IsInΓ0+IsInΓ+Code, def:Code.Symbols, def:Code.Term}
    Given terms $\sigma, \tau$, the codes of atomic formulas are
    $$
    \ulcorner{\sigma = \tau}\urcorner = 
    \langle\ulcorner{=}\urcorner, \ulcorner{\sigma}\urcorner, \ulcorner{\tau}\urcorner \rangle
    \quad \text{and} \quad
    \ulcorner{\sigma \in \tau}\urcorner = 
    \langle\ulcorner{\in}\urcorner, \ulcorner{\sigma}\urcorner, \ulcorner{\tau}\urcorner \rangle.
    $$
    Non-atomic formulas follow the inductive definition
    $$
    \begin{aligned}
        \ulcorner{\neg\varphi}\urcorner & = 
        \langle\ulcorner{\neg}\urcorner, \ulcorner{\varphi}\urcorner \rangle, \\
        \ulcorner{\varphi \lor \psi}\urcorner & = 
        \langle\ulcorner{\lor}\urcorner, \ulcorner{\varphi}\urcorner, 
        \ulcorner{\psi}\urcorner \rangle, \\
        \ulcorner{\exists x_k \varphi}\urcorner & = 
        \langle\ulcorner{\exists}\urcorner, \ulcorner{x_k}\urcorner, 
        \ulcorner{\varphi}\urcorner \rangle.
    \end{aligned}
    $$
\end{definition}

