\chapter{The Provability Formula}

For any $\mathcal{L}$-formula $\varphi$, we now aim to define a condition applicable to
its code $\ulcorner\varphi\urcorner$ that holds true if and only if $\varphi$ is a theorem of HF, 
i.e. $\varphi$ is provable. 
Specifically, we seek to identify an $\mathcal{L}$-formula $\operatorname{Pf}(x)$,
referred to as the \textit{provability formula}, such that
$$
\vdash \varphi \,\text{ if and only if } \vdash \operatorname{Pf}(\ulcorner{\varphi}\urcorner).
$$
To establish this equivalence, we shall invoke $\mathfrak{S}$, the standard model of HF,
by ensuring that $\vdash \operatorname{Pf}(\ulcorner{\varphi}\urcorner)$ if and only if
$\mathfrak{S} \vDash \operatorname{Pf}(\ulcorner{\varphi}\urcorner)$. 
Then it remains to define $\operatorname{Pf}(x)$ in such a way that
$$
\vdash \varphi \,\text{ if and only if } 
\mathfrak{S} \vDash \operatorname{Pf}(\ulcorner{\varphi}\urcorner),
$$
a process that is tedious but straightforward once the appropriate definitions and results 
are in place.

By soundness, we have that $\vdash \varphi$ implies 
$\mathfrak{S} \vDash \varphi$ for any $\mathcal{L}$-formula $\varphi$.
However, the converse does not hold in general, with the sentence in Gödel's first incompleteness
theorem as an example.
Fortunately, there is a substantial class of sentences for which being modelled by $\mathfrak{S}$
is equivalent to being a theorem of HF.
We call these sentences $\Sigma$-sentences, and it follows that, given an $\mathcal{L}$-formula 
$\varphi$, the provability formula $\operatorname{Pf}(\ulcorner{\varphi}\urcorner)$ should be
a $\Sigma$-sentence.

\section{Sigma-formulas}
%\section{\(\Sigma\)-formulas}

To begin, we introduce strict $\Sigma$-formulas, 
which are $\mathcal{L}$-formulas that only include bounded universal quantifiers, variables, 
and the symbols $\in, \lor, \land, \exists$.

\begin{definition}[Strict $\Sigma$-formula]
\label{def:IsInSSigma}
\lean{HF.IsInSSigma}
\leanok
\uses{def:Lang}
The class $\Sigma$ of \textit{strict $\Sigma$-formulas} is the smallest class of 
$\mathcal{L}$-formulas such that
\begin{enumerate}
    \item The atomic formula $x_i \in x_j$ is in $\Sigma$ for all variables $x_i, x_j$.
    \item If $\varphi, \psi$ are in $\Sigma$, then so are 
        $\varphi \lor \psi$ and $\varphi \land \psi$.
    \item If $\varphi$ is in $\Sigma$, then so are $\exists x_i \varphi$ and 
        $\forall (x_i \in x_j) \varphi$ for all distinct variables $x_i, x_j$.
\end{enumerate}
\end{definition}

\begin{definition}[$\Sigma$-formula]
    \label{def:IsSigma}
    \lean{HF.IsSigma}
    \leanok
    \uses{def:IsInSSigma}
    Say an $\mathcal{L}$-formula $\varphi$ is a \textit{$\Sigma$-formula} 
    if there exists a strict $\Sigma$-formula $\psi$ such that 
    $\vdash \varphi \leftrightarrow \psi$.
\end{definition}

\begin{lemma}
    \lean{HF.subset_isSigma, HF.eq_isSigma, HF.enlarge_isSigma, HF.eq_empty_isSigma, 
        HF.mem_empty_isSigma, HF.empty_mem_isSigma}
    \label{lem:subset_isSigma+...+empty_mem_isSigma}
    \leanok
    \uses{thm:completeness, lem:notin_empty+mem_enlarge+mem_enlarge_empty, thm:exten_prop, 
        def:Subset+SSubset, cor:mem_irrefl, def:IsSigma}
    The $\mathcal{L}$-formulas $x \subseteq y$, $x = y$, $z = x \lhd y$, $x=0$, $x\in 0$,
    $0 \in x$ and all atomic formulas are $\Sigma$-formulas.
\end{lemma}

\begin{proof}
    The following are theorems of HF:
    \begin{enumerate}
        \item $x \subseteq y \leftrightarrow \forall (u \in x)[u \in y]$.
        \item $x = y \leftrightarrow x \subseteq y \land y \subseteq x$.
        \item $z = x \lhd y \leftrightarrow \forall(u \in z)[u \in x \lor u = y] 
        \land (x \subseteq z) \land (y \in z)$.
        \item $x = 0 \leftrightarrow \forall (u \in x)[u \in u]$ (see Lemma 
        \ref{lem:notin_empty+mem_enlarge+mem_enlarge_empty} and Corollary \ref{cor:mem_irrefl}).
        \item $x \in 0 \leftrightarrow x \in x$ (see Corollary \ref{cor:mem_irrefl}).
        \item $0 \in x \leftrightarrow \exists y (y = 0 \land y \in x)$.
    \end{enumerate}
    It follows from Definition \ref{def:IsInSSigma}.1, 
    Lemma \ref{lem:subset_isSigma+...+empty_mem_isSigma}.2 and 
    Lemma \ref{lem:subset_isSigma+...+empty_mem_isSigma}.3 that every atomic formula is a
    $\Sigma$-formula.
\end{proof}

\begin{lemma}
    \label{lem:Ord_isSigma+Seq_isSigma+LstSeq_isSigma}
    \uses{def:IsTrans+IsOrd, def:IsSeq, def:IsSigma, lem:subset_isSigma+...+empty_mem_isSigma}
    Introduce the $\mathcal{L}$-formula $\operatorname{LstSeq}(s,k,y)$, which means
    $\operatorname{Seq}(s,k)$ and $s$ terminates with $y$.
    
    The $\mathcal{L}$-formulas $\operatorname{Ord}(x)$, $\operatorname{Seq}(s,k)$ and
    $\operatorname{LstSeq}(s,k,y)$ are $\Sigma$-formulas.
\end{lemma}

\begin{proof}
    The following are theorems of HF:
    \begin{enumerate}
        \item $\operatorname{Ord}(x) \leftrightarrow 
        \forall(y \in x)\{[y \subseteq x] \land \forall(z \in y)[z \subseteq y]\}$ (see 
        Definition \ref{def:IsTrans+IsOrd} and 
        Lemma \ref{lem:subset_isSigma+...+empty_mem_isSigma}).
        \item $\operatorname{Seq}(s, k) \leftrightarrow \operatorname{Ord}(k) 
            \land \varnothing \in k \land \forall(n \in k) \exists u[\langle n, u\rangle \in s] 
            \land$
            $\forall(y, z \in s) \exists(m, n \in k) \exists u \exists v[y=\langle m, u\rangle 
            \land z=\langle n, v\rangle \land(m \neq n \lor u=v)]$.
        \item $\operatorname{LstSeq}(s, k, y) \leftrightarrow \operatorname{Seq}(s, k) 
            \land \exists n(n \lhd n=k \land\langle n, y\rangle \in s)$ 
            (see Lemma \ref{lem:subset_isSigma+...+empty_mem_isSigma}).
    \end{enumerate}
    Observe that, in 2., $m \neq n$ is a $\Sigma$-formula if $m,n$ are ordinals, as
    it is equivalent to $m \in n \lor n \in m$ 
    (see Corollary \ref{cor:IsOrd.exclusive_comparability+...+IsOrd.succ_inj}).
\end{proof}

A \textit{$\Sigma$-sentence} is a $\Sigma$-formula that is a sentence, 
i.e. it has no free variables. 
As discussed earlier, the key property of $\Sigma$-sentences is that 
being modelled by $\mathfrak{S}$ is equivalent to being a theorem of HF.
In order to establish this, we need the class of $\Sigma_{\lhd}$-formulas.

\begin{definition}
    \label{def:IsSigma_lhd}
    \uses{def:Lang, def:IsSigma}
    Say an $\mathcal{L}$-formula is a \textit{$\Sigma_{\lhd}$-formula} if it is made up
    of terms, the symbols ${\varnothing}$, ${=}$, ${\in}$, ${\lor}$, ${\land}$, ${\exists x_i}$,
    and ${\forall(x_i \in \tau)}$, where $\tau$ is any term that does not contain $x_i$.

    For a $\Sigma_{\lhd}$-formula $\alpha$, its length $\lambda(\alpha)$ is defined to be 
    the total number of occurrences of the symbols $\lor$, $\land$, $\exists$, and $\forall$.
\end{definition}

\begin{theorem}
    \label{thm:stdModel.prf_iff_models}
    \lean{HF.stdModel.prf_iff_models}
    \uses{thm:completeness, def:C, lem:C.exists_finset_shorter_and_mem_iff_iSup, 
        def:stdModel, thm:stdModel.model_of_consistent, def:IsSigma, def:IsSigma_lhd}
    If HF is consistent, then, for every $\Sigma$-sentence $\alpha$, 
    $$
    \vdash \alpha \,\text{ if and only if } 
    \mathfrak{S} \vDash \alpha.
    $$
\end{theorem}

\begin{proof}
    By consistency, $\mathfrak{S}$ is a model of HF.
    Then, by soundness, it remains to show that for every $\Sigma$-sentence $\alpha$,
    $\mathfrak{S} \vDash \alpha$ implies $\vdash \alpha$.
    Furthermore, since every strict $\Sigma$-sentence is a $\Sigma_{\lhd}$-sentence, 
    it suffices to prove the result for every $\Sigma_{\lhd}$-sentence $\alpha$.
    We proceed by induction on the length $\lambda (\alpha)$.

    Suppose $\lambda(\alpha) = 0$. Then $\alpha$ must be of the form $\sigma \in \tau$ or 
    $\sigma = \tau$ for some constant terms $\sigma, \tau$.
    The result follows directly from the definition of ${\in}$ and ${=}$ in $\mathfrak{S}$,
    i.e. in Definition \ref{def:stdModel}.

    Now assume $\lambda(\alpha) > 0$ and suppose the result holds for all $\Sigma_{\lhd}$-sentences 
    of length less than $\lambda (\alpha)$.
    We consider the following cases:
    \begin{itemize}
        \item $\alpha$ is $\beta \lor \gamma$. By the inductive assumption,
        \begin{equation*}
            \mathfrak{S} \vDash \alpha \rightarrow 
            (\mathfrak{S} \vDash \beta \text{ or } \mathfrak{S} \vDash \gamma) \rightarrow
            (\vdash \beta \text{ or } \vdash \gamma) \rightarrow 
            \vdash \beta \lor \gamma.
        \end{equation*}
        \item $\alpha$ is $\beta \land \gamma$. Analogous to the previous case.
        \item $\alpha$ is $\exists x \beta (x)$.
        If $\mathfrak{S} \vDash \alpha$, then there exists a term $\tau$ such that
        $\mathfrak{S} \vDash \beta(\tau)$. 
        Since $\lambda(\beta(\tau)) = \lambda(\beta) < \lambda(\alpha)$, it follows from
        the inductive assumption that $\vdash \beta(\tau)$, 
        and hence $\vdash \exists x \beta (x)$.
        \item $\alpha$ is $\forall (x \in \tau) \beta(x)$, where $\tau$ is a term that does not
        contain $x$. Since there are no free variables in $\alpha$, $\tau$ must be a constant term.
        By Lemma \ref{lem:C.exists_finset_shorter_and_mem_iff_iSup}, there are 
        $\tau_1, \ldots, \tau_m \in \mathbb{C}$ such that
        $$
        \vdash \forall (x \in \tau) \beta(x) \leftrightarrow
        \beta(\tau_1) \land \ldots \land \beta(\tau_m).
        $$
        Since $\lambda(\beta(\tau_j)) = \lambda(\beta(x)) < \lambda(\alpha)$, it follows 
        from the inductive assumption that
        \begin{equation*}
            \begin{split}
                & \mathfrak{S} \vDash \alpha \rightarrow 
                \mathfrak{S} \vDash \beta(\tau_1) \text{ and } \ldots \text{ and } 
                \mathfrak{S} \vDash \beta(\tau_m) \rightarrow \\
                &\vdash \beta(\tau_1) \text{ and } \ldots \text{ and } \vdash \beta(\tau_m) 
                \rightarrow
                \vdash \beta(\tau_1) \land \ldots \land \beta(\tau_m) \rightarrow
                \vdash \forall (x \in \tau) \beta(x).
            \end{split}
        \end{equation*}
    \end{itemize}
\end{proof}

\section{The formula Pf}

In this section, we summarise the the more detailed work required to construct the 
$\Sigma$-formula $\operatorname{Pf}(x)$.
Ultimately, we want the latter to \textit{mean} that $x$ codes an $\mathcal{L}$-formula 
that is a theorem of HF.
Here, by 'to mean' we imply the equivalence between $\vdash \operatorname{Pf}(x)$ and
$x$ coding a theorem of HF.

The remaining work of establishing $\operatorname{Pf}(x)$ essentially boils down to verifying 
that all fundamental relationships between variables, terms, and formulas that 
underlie a proof in HF can be expressed by a $\Sigma$-formula that only involves codes as arguments.
By the latter we mean that the $\Sigma$-formula is expressed in such a way that the arguments 
necessarily be codes if the $\Sigma$-formula is a theorem of HF.
In the following definitions, 
we will present the abbreviations and the \textit{meanings} of all these relevant 
$\Sigma$-formulas, which are sub-formulas of $\operatorname{Pf}(x)$.
For the corresponding written-out $\Sigma$-formulas themselves, the reader is
referred to Chapter 3 and 4 of \cite{swierczkowski2003finite}, where they are described in detail.

For the sake of readability, the definitions of the relevant $\Sigma$-formulas are presented
as lemmas.

\begin{lemma}
    \label{lem:IsSigma.Relations}
    \uses{def:Lang, def:IsTrans+IsOrd, def:IsSeq, def:Code.Term, def:Code.Formula, def:IsSigma, 
        lem:Ord_isSigma+Seq_isSigma+LstSeq_isSigma}
    There exist 25 $\Sigma$-formulas, solely involving codes as arguments, that express certain
    relationships between variables, terms, and formulas.
\end{lemma}

\begin{proof}
    Introduce the following abbreviations of $\Sigma$-formulas:
    \begin{enumerate}
        \item $\operatorname{Var}(x)$\\
        Means: $x$ is the code of a variable.
        \item $\operatorname{SeqTerm}(s,k,t)$\\
        Means: $s$ is a sequence of length $k$ and $s$ terminates with the code $t$ of a term.
        \item $\operatorname{Term}(t)$\\
        Means: $t$ is the code of a term.
        \item $\operatorname{Const}(t)$\\
        Means: $t$ is the code of a constant term.
        \item $\operatorname{NecSeqTerm}(s,k,t)$\\
        Means: $s$ is a sequence of length $k$, $s$ terminates with the code $t$ of a term and
        all terms whose codes appear in $s$ are sub-terms of the term coded by $t$.
        \item $\operatorname{VarOccTerm}(v,t)$\\
        Means: $v$ codes a variable, $t$ codes a term and 
        the variable coded by $v$ occurs in the term coded by $t$.
        \item $\operatorname{VarOccTerm}(v,t)$\\
        Means: $v$ codes a variable, $t$ codes a term and 
        the variable coded by $v$ does not occur in the term coded by $t$.
        \item $\operatorname{At}(y)$\\
        Means: $y$ is the code of an atomic formula.
        \item $\operatorname{VarOccAt}(v,y)$\\
        Means: $v$ codes a variable, $y$ codes an atomic formula and 
        the variable coded by $v$ occurs in the atomic formula coded by $y$.
        \item $\operatorname{VarNonOccAt}(v,y)$\\
        Means: $v$ codes a variable, $y$ codes an atomic formula and 
        the variable coded by $v$ does not occur in the atomic formula coded by $y$.
        \item $\operatorname{MakeForm}(u,w,v,y)$\\
        Means: $v$ codes a variable and if $u,w$ are codes of $\mathcal{L}$-formulas, 
        then $y$ is the code of an $\mathcal{L}$-formula created from one or 
        both $\mathcal{L}$-formulas coded by $u$, $w$ and possibly the variable coded by $v$ 
        by a single application of ${\lor}$, ${\neg}$ or ${\exists}$.
        \item $\operatorname{SeqForm}(s,k,y)$\\
        Means: $s$ is a sequence of length $k$ and $s$ terminates with the code $y$ of a 
        $\mathcal{L}$-formula.
        \item $\operatorname{Form}(y)$\\
        Means: $y$ is the code of an $\mathcal{L}$-formula.
        \item $\operatorname{NonAt}(y)$\\
        Means: $y$ is the code of a non-atomic formula.
        \item $\operatorname{VarTopForm}(v,y)$\\
        Means: $y$ is the code of an $\mathcal{L}$-formula beginning with $\exists x_i$,
        where $v$ codes the variable $x_i$.
        \item $\operatorname{SeqVarTopForm}(v,z,n,y)$\\
        Means: $y$ is the code of an $\mathcal{L}$-formula and $z$ is a sequence of length $n$ 
        showing how the $\mathcal{L}$-formula coded by $y$ is made up of $\mathcal{L}$-formulas 
        beginning with $\exists x_i$ — where $v$ codes the variable $x_i$ — and of atomic formulas,
        by using ${\lor}$, ${\neg}$ and ${\exists x_j}$, where $j\neq i$.
        \item $\operatorname{NecSeqVarTopForm}(v,z,n,y)$\\
        Means: as above, but additionally all $\mathcal{L}$-formulas whose codes appear in $z$
        are sub-formulas of the $\mathcal{L}$-formula coded by $y$.
        \item $\operatorname{VarOccFreeForm}(v,y)$\\
        Means: $v$ codes a variable, $y$ codes an $\mathcal{L}$-formula and the variable coded by
        $v$ occurs freely in the $\mathcal{L}$-formula coded by $y$.
        \item $\operatorname{VarNonOccFreeForm}(v,y)$\\
        Means: $v$ codes a variable, $y$ codes an $\mathcal{L}$-formula and the variable coded by
        $v$ does not occur freely in the $\mathcal{L}$-formula coded by $y$.
        \item $\operatorname{TermsSubsVarForm}(t,v,y)$\\
        Means: $t$ codes a term, $v$ codes a variable, $y$ codes an $\mathcal{L}$-formula, and
        the term coded by $t$ is a substitutable for the variable coded by $v$ in the 
        $\mathcal{L}$-formula coded by $y$.
        \item $\operatorname{SeqRepVarTermTerm}(s, s^\prime, k, v, t, u, u^\prime)$\\
        Means: $v$ codes a variable, $t$ codes a term, $u$ and $u^\prime$ are coding terms 
        and $s$ and $s^\prime$ are sequences of length $k$ such that $s$ terminates with $u$ 
        and $s^\prime$ terminates with $u^\prime$ and the term coded by $u^\prime$ is obtained from 
        that coded by $u$ by replacing each occurrence of the variable coded by $v$ by the term 
        coded by $t$.
        \item $\operatorname{RepVarTermTerm}(v, t, u, u^\prime)$\\
        Means: $v$ codes a variable, $t$ codes a term, $u$ and $u^\prime$ code terms and 
        replacing in the term coded by $u$ the variable coded by $v$, at all of its occurrences, 
        by the term coded by $t$ one obtains the term coded by $u^\prime$.
        \item $\operatorname{RepVarTermAt}(v, t, y, y^\prime)$\\
        Means: $v$ codes a variable, $t$ codes a term, $y$ and $y^\prime$ are coding atomic formulas 
        and replacing the variable coded by $v$ by the term coded by $t$, 
        at all occurrences of this variable in the atomic formula coded by $y$, 
        one obtains the atomic formula coded by $y^\prime$.
        \item $\operatorname{SeqRepVarTermForm}(s, s^\prime, k, v, t, y, y^\prime)$\\
        Means: $v$ codes a variable, $t$ codes a term, $y$ and $y^\prime$ code 
        $\mathcal{L}$-formulas and $s$ and $s^\prime$ are sequences of length $k$ such that $s$ 
        terminates with $y$ and $s^\prime$ terminates with $y^\prime$ and the $\mathcal{L}$-formula
        coded by $y^\prime$ is obtained from that coded by $y$ by replacing each free occurrence of 
        the variable coded by $v$ by the term coded by $t$.
        \item $\operatorname{RepVarTermForm}(v, t, y, y^\prime)$\\
        Means: $v$ codes a variable, $t$ codes a term, $y$ and $y^\prime$ code atomic 
        $\mathcal{L}$-formulas and replacing the variable coded by $v$ by the term coded by $t$, 
        at all free occurrences of this variable in the $\mathcal{L}$-formula coded by $y$, 
        one obtains the $\mathcal{L}$-formula coded by $y^\prime$.
    \end{enumerate}
\end{proof}

With these $\Sigma$-formulas in place, 
we can now construct the provability formula $\operatorname{Pf}(x)$.
To that end, we need eight other $\Sigma$-formulas that express arguments being 
a code of an axiom or a deduction rule, which directly yield $\operatorname{Pf}(x)$.

\begin{lemma}[The formula Pf]
    \label{lem:IsSigma.Ax+...+IsSigma.Pf}
    \lean{HF.IsSigma.Pf}
    \uses{def:Lang, def:Axioms, def:Bool.Axioms, def:Spec.Axiom, def:Equality.Axioms,
        def:prf.MP+prf.Subst+prf.ExIntro, def:Code.Formula, 
        def:IsSigma, lem:IsSigma.Relations}
    There exist eight $\Sigma$-formulas, solely involving codes as arguments, that, together with
    the $\Sigma$-formulas in Lemma \ref{lem:IsSigma.Relations}, are sub-formulas of
    the $\Sigma$-formula $\operatorname{Pf}(x)$.
\end{lemma}

\begin{proof}
    Introduce the following abbreviations of $\Sigma$-formulas:
    \begin{enumerate}
        \item $\operatorname{Ax}(y)$\\
        Means: $y$ is the code of the first axiom of HF, the second axiom of HF, or 
        an equality axiom 
        (see Definition \ref{def:Axioms} and Definition \ref{def:Equality.Axioms}).
        \item $\operatorname{Ind}(x)$\\
        Means: $y$ is the code of an instance of the induction scheme.
        \item $\operatorname{Sent}(x)$\\
        Means: $x$ is the code of an $\mathcal{L}$-formula that is an instance 
        of one of the five sentential axioms (see Definition \ref{def:Bool.Axioms}).
        \item $\operatorname{Spec}(x)$\\
        Means: $x$ is the code of a specialisation axiom (see Definition \ref{def:Spec.Axiom}).
        \item $\operatorname{ModPon}(x,y,y^\prime)$\\
        Means: If $x,y,y^\prime$ are codes of $\mathcal{L}$-formulas, 
        then $y^\prime$ codes an $\mathcal{L}$-formula which follows from the $\mathcal{L}$-formulas
        coded by $x$ and $y$ by an application of the Modus Ponens rule 
        (see Definition \ref{def:prf.MP+prf.Subst+prf.ExIntro}).
        \item $\operatorname{Subst}(y,y^\prime)$\\
        Means: $y,y^\prime$ are codes of $\mathcal{L}$-formulas and
        the $\mathcal{L}$-formula coded by $y^\prime$ is obtained from that coded by $y$ by
        an application of the substitution rule.
        \item $\operatorname{ExIntro}(y, y^\prime)$\\
        Means: $y,y^\prime$ are codes of $\mathcal{L}$-formulas and
        the $\mathcal{L}$-formula coded by $y^\prime$ is obtained from that coded by $y$ by
        an application of the $\exists$-introduction rule.
        \item $\operatorname{Prf}(s, k, y)$\\
        Means: $s$ is a sequence of length $k$ of codes of $\mathcal{L}$-formulas 
        such that the corresponding sequence of formulas is a proof of the 
        $\mathcal{L}$-formula coded by $y$.
        \item $\operatorname{Pf}(y)$\\
        Means: $y$ is the code of an $\mathcal{L}$-formula that is a theorem of HF.
    \end{enumerate}
\end{proof}

\begin{theorem}[Proof formalisation condition]
    \label{thm:prf_iff_prf_Pf_code}
    \lean{HF.prf_iff_prf_Pf_code}
    \leanok
    \uses{def:Code.Formula, thm:stdModel.prf_iff_models, lem:IsSigma.Relations, 
        lem:IsSigma.Ax+...+IsSigma.Pf}
    If HF is consistent, then, for any $\mathcal{L}$-formula $\varphi$,
    $$
    \vdash \varphi \,\text{ if and only if } \vdash \operatorname{Pf}(\ulcorner{\varphi}\urcorner).
    $$
\end{theorem}

\begin{proof}
    Assume HF is consistent. First, suppose $\vdash \varphi$.
    Then there is a finite sequence of $\mathcal{L}$-formulas, $\varphi_{\varnothing}, \ldots, 
    \varphi_p$, that forms a proof of $\varphi$.
    If $s = \{\langle l, \ulcorner{\varphi_l}\urcorner \rangle : l \in \operatorname{succ}(p) \}$,
    then $s \in \mathbb{S}$ (the domain of $\mathfrak{S}$, see Definition \ref{def:stdModel}) and
    $\mathfrak{S} \vDash \operatorname{Prf}(s, \operatorname{succ}(p), 
    \ulcorner{\varphi}\urcorner)$, 
    so that $\mathfrak{S} \vDash \operatorname{Pf}(\ulcorner{\varphi}\urcorner)$.
    Since $\operatorname{Pf}(\ulcorner{\varphi}\urcorner)$ is a $\Sigma$-sentence,
    by Theorem \ref{thm:stdModel.prf_iff_models}, 
    $\vdash \operatorname{Pf}(\ulcorner{\varphi}\urcorner)$.
    
    Now suppose $\vdash \operatorname{Pf}(\ulcorner{\varphi}\urcorner)$.
    Then, by Theorem \ref{thm:stdModel.prf_iff_models}, 
    $\mathfrak{S} \vDash \operatorname{Pf}(\ulcorner{\varphi}\urcorner)$.
    This means there exists a sequence $s \in \mathbb{S}$ of length $k$,
    where each $s_l$ codes some $\mathcal{L}$-formula $\phi_l$, 
    so that $\phi_{\operatorname{pred}(k)}$ equals $\varphi$ and the sequence
    $\phi_{\varnothing}, \ldots, \phi_{\operatorname{pred}(k)}$ is a proof of $\varphi$.
    Thus $\vdash \varphi$.
\end{proof}