\chapter{The Provability Formula}

For any $\mathcal{L}$-formula $\varphi$, we now aim to define a condition applicable to
its code $\ulcorner\varphi\urcorner$ that holds true if and only if $\varphi$ is a theorem of HF, 
i.e. $\varphi$ is provable. 
Specifically, we seek to identify an $\mathcal{L}$-formula $\operatorname{Pf}(x)$,
referred to as the \textit{provability formula}, such that
$$
\vdash \varphi \,\text{ if and only if } \vdash \operatorname{Pf}(\ulcorner{\varphi}\urcorner).
$$
To establish this equivalence, we shall invoke $\mathfrak{S}$, the standard model of HF,
by ensuring that $\vdash \operatorname{Pf}(\ulcorner{\varphi}\urcorner)$ if and only if
$\mathfrak{S} \models \operatorname{Pf}(\ulcorner{\varphi}\urcorner)$. 
Then it remains to define $\operatorname{Pf}(x)$ in such a way that
$$
\vdash \varphi \,\text{ if and only if } 
\mathfrak{S} \models \operatorname{Pf}(\ulcorner{\varphi}\urcorner),
$$
a process that is tedious but straightforward once the appropriate definitions and results 
are in place.

By soundness, we have that $\vdash \varphi$ implies 
$\mathfrak{S} \models \varphi$ for any $\mathcal{L}$-formula $\varphi$.
However, the converse does not hold in general, with the sentence in Gödel's first incompleteness
theorem as an example.
Fortunately, there is a substantial class of sentences for which being modelled by $\mathfrak{S}$
is equivalent to being a theorem of HF.
We call these sentences $\Sigma$-sentences, and it follows that, given an $\mathcal{L}$-formula 
$\varphi$, the provability formula $\operatorname{Pf}(\ulcorner{\varphi}\urcorner)$ should be
a $\Sigma$-sentence.

\section{Sigma-formulas}
%\section{\(\Sigma\)-formulas}

To begin, we introduce strict $\Sigma$-formulas, 
which are $\mathcal{L}$-formulas that only include bounded universal quantifiers, variables, 
and the symbols $\in, \lor, \land, \exists$.

\begin{definition}[Strict $\Sigma$-formula]
\label{def:IsInSSigma}
\lean{HF.IsInSSigma}
\leanok
\uses{def:Lang}
The class $\Sigma$ of \textit{strict $\Sigma$-formulas} is the smallest class of 
$\mathcal{L}$-formulas such that
\begin{enumerate}
    \item The atomic formula $x_i \in x_j$ is in $\Sigma$ for all variables $x_i, x_j$.
    \item If $\varphi, \psi$ are in $\Sigma$, then so are 
        $\varphi \lor \psi$ and $\varphi \land \psi$.
    \item If $\varphi$ is in $\Sigma$, then so are $\exists x_i \varphi$ and 
        $\forall (x_i \in x_j) \varphi$ for all distinct variables $x_i, x_j$.
\end{enumerate}
\end{definition}

\begin{definition}[$\Sigma$-formula]
    \label{def:IsSigma}
    \lean{HF.IsSigma}
    \leanok
    \uses{def:IsInSSigma}
    Say an $\mathcal{L}$-formula $\varphi$ is a \textit{$\Sigma$-formula} 
    if there exists a strict $\Sigma$-formula $\psi$ such that 
    $\vdash \varphi \leftrightarrow \psi$.
\end{definition}


