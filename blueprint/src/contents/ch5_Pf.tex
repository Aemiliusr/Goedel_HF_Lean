\chapter{The Provability Formula}

For any $\mathcal{L}$-formula $\varphi$, we now aim to define a condition applicable to
its code $\ulcorner\varphi\urcorner$ that holds true if and only if $\varphi$ is a theorem of HF, 
i.e. $\varphi$ is provable. 
Specifically, we seek to identify an $\mathcal{L}$-formula $\operatorname{Pf}(x)$,
referred to as the \textit{provability formula}, such that
$$
\vdash \varphi \,\text{ if and only if } \vdash \operatorname{Pf}(\ulcorner{\varphi}\urcorner).
$$
To establish this equivalence, we shall invoke $\mathfrak{S}$, the standard model of HF,
by ensuring that $\vdash \operatorname{Pf}(\ulcorner{\varphi}\urcorner)$ if and only if
$\mathfrak{S} \vDash \operatorname{Pf}(\ulcorner{\varphi}\urcorner)$. 
Then it remains to define $\operatorname{Pf}(x)$ in such a way that
$$
\vdash \varphi \,\text{ if and only if } 
\mathfrak{S} \vDash \operatorname{Pf}(\ulcorner{\varphi}\urcorner),
$$
a process that is tedious but straightforward once the appropriate definitions and results 
are in place.

By soundness, we have that $\vdash \varphi$ implies 
$\mathfrak{S} \vDash \varphi$ for any $\mathcal{L}$-formula $\varphi$.
However, the converse does not hold in general, with the sentence in Gödel's first incompleteness
theorem as an example.
Fortunately, there is a substantial class of sentences for which being modelled by $\mathfrak{S}$
is equivalent to being a theorem of HF.
We call these sentences $\Sigma$-sentences, and it follows that, given an $\mathcal{L}$-formula 
$\varphi$, the provability formula $\operatorname{Pf}(\ulcorner{\varphi}\urcorner)$ should be
a $\Sigma$-sentence.

\section{Sigma-formulas}
%\section{\(\Sigma\)-formulas}

To begin, we introduce strict $\Sigma$-formulas, 
which are $\mathcal{L}$-formulas that only include bounded universal quantifiers, variables, 
and the symbols $\in, \lor, \land, \exists$.

\begin{definition}[Strict $\Sigma$-formula]
\label{def:IsInSSigma}
\lean{HF.IsInSSigma}
\leanok
\uses{def:Lang}
The class $\Sigma$ of \textit{strict $\Sigma$-formulas} is the smallest class of 
$\mathcal{L}$-formulas such that
\begin{enumerate}
    \item The atomic formula $x_i \in x_j$ is in $\Sigma$ for all variables $x_i, x_j$.
    \item If $\varphi, \psi$ are in $\Sigma$, then so are 
        $\varphi \lor \psi$ and $\varphi \land \psi$.
    \item If $\varphi$ is in $\Sigma$, then so are $\exists x_i \varphi$ and 
        $\forall (x_i \in x_j) \varphi$ for all distinct variables $x_i, x_j$.
\end{enumerate}
\end{definition}

\begin{definition}[$\Sigma$-formula]
    \label{def:IsSigma}
    \lean{HF.IsSigma}
    \leanok
    \uses{def:IsInSSigma}
    Say an $\mathcal{L}$-formula $\varphi$ is a \textit{$\Sigma$-formula} 
    if there exists a strict $\Sigma$-formula $\psi$ such that 
    $\vdash \varphi \leftrightarrow \psi$.
\end{definition}

\begin{lemma}
    \label{lem:subset_isSigma+...+empty_mem_isSigma}
    \uses{lem:notin_empty+mem_enlarge+mem_enlarge_empty, thm:exten_prop, def:Subset+SSubset, 
        cor:mem_irrefl, def:IsSigma}
    The $\mathcal{L}$-formulas $x \subseteq y$, $x = y$, $z = x \lhd y$, $x=0$, $x\in 0$,
    $0 \in x$ and all atomic formulas are $\Sigma$-formulas.
\end{lemma}

\begin{proof}
    The following are theorems of HF:
    \begin{enumerate}
        \item $x \subseteq y \leftrightarrow \forall (u \in x)[u \in y]$.
        \item $x = y \leftrightarrow x \subseteq y \land y \subseteq x$.
        \item $z = x \lhd y \leftrightarrow \forall(u \in z)[u \in x \lor u = y] 
        \land (x \subseteq z) \land (y \in z)$.
        \item $x = 0 \leftrightarrow \forall (u \in x)[u \in u]$ (see Lemma 
        \ref{lem:notin_empty+mem_enlarge+mem_enlarge_empty} and Corollary \ref{cor:mem_irrefl}).
        \item $x \in 0 \leftrightarrow x \in x$ (see Corollary \ref{cor:mem_irrefl}).
        \item $0 \in x \leftrightarrow \exists y (y = 0 \land y \in x)$.
    \end{enumerate}
    It follows from Definition \ref{def:IsInSSigma}.1, 
    Lemma \ref{lem:subset_isSigma+...+empty_mem_isSigma}.2 and 
    Lemma \ref{lem:subset_isSigma+...+empty_mem_isSigma}.3 that every atomic formula is a
    $\Sigma$-formula.
\end{proof}

\begin{lemma}
    \label{lem:Ord_isSigma+Seq_isSigma+LstSeq_isSigma}
    \uses{def:IsTrans+IsOrd, def:IsSeq, def:IsSigma, lem:subset_isSigma+...+empty_mem_isSigma}
    Introduce the $\mathcal{L}$-formula $\operatorname{LstSeq}(s,k,y)$, which means
    $\operatorname{Seq}(s,k)$ and $s$ terminates with $y$.
    
    The $\mathcal{L}$-formulas $\operatorname{Ord}(x)$, $\operatorname{Seq}(s,k)$ and
    $\operatorname{LstSeq}(s,k,y)$ are $\Sigma$-formulas.
\end{lemma}

\begin{proof}
    The following are theorems of HF:
    \begin{enumerate}
        \item $\operatorname{Ord}(x) \leftrightarrow 
        \forall(y \in x)\{[y \subseteq x] \land \forall(z \in y)[z \subseteq y]\}$ (see 
        Definition \ref{def:IsTrans+IsOrd} and 
        Lemma \ref{lem:subset_isSigma+...+empty_mem_isSigma}).
        \item $\operatorname{Seq}(s, k) \leftrightarrow \operatorname{Ord}(k) 
            \land \varnothing \in k \land \forall(n \in k) \exists u[\langle n, u\rangle \in s] 
            \land$
            $\forall(y, z \in s) \exists(m, n \in k) \exists u \exists v[y=\langle m, u\rangle 
            \land z=\langle n, v\rangle \land(m \neq n \lor u=v)]$.
        \item $\operatorname{LstSeq}(s, k, y) \leftrightarrow \operatorname{Seq}(s, k) 
            \land \exists n(n \lhd n=k \land\langle n, y\rangle \in s)$ 
            (see Lemma \ref{lem:subset_isSigma+...+empty_mem_isSigma}).
    \end{enumerate}
    Observe that, in 2., $m \neq n$ is a $\Sigma$-formula if $m,n$ are ordinals, as
    it is equivalent to $m \in n \lor n \in m$ 
    (see Corollary \ref{cor:IsOrd.exclusive_comparability+...+IsOrd.succ_inj}).
\end{proof}

A \textit{$\Sigma$-sentence} is a $\Sigma$-formula that is a sentence, 
i.e. it has no free variables. 
As discussed earlier, the key property of $\Sigma$-sentences is that 
being modelled by $\mathfrak{S}$ is equivalent to being a theorem of HF.
In order to establish this, we need the class of $\Sigma_{\lhd}$-formulas.

\begin{definition}
    \label{def:IsSigma_lhd}
    \uses{def:Lang, def:IsSigma}
    Say an $\mathcal{L}$-formula is a \textit{$\Sigma_{\lhd}$-formula} if it is made up
    of terms, the symbols ${\varnothing}$, ${=}$, ${\in}$, ${\lor}$, ${\land}$, ${\exists x_i}$,
    and ${\forall(x_i \in \tau)}$, where $\tau$ is any term that does not contain $x_i$.

    For a $\Sigma_{\lhd}$-formula $\alpha$, its length $\lambda(\alpha)$ is defined to be 
    the total number of occurrences of the symbols $\lor$, $\land$, $\exists$, and $\forall$.
\end{definition}

\begin{theorem}
    \label{thm:stdModel.prf_iff_models}
    \uses{def:IsSigma_lhd}
    For every $\Sigma$-sentence $\alpha$, 
    $$
    \vdash \alpha \,\text{ if and only if } 
    \mathfrak{S} \vDash \alpha.
    $$
\end{theorem}

\begin{proof}
    By soundness, it remains to show that for every $\Sigma$-sentence $\alpha$,
    $\mathfrak{S} \vDash \alpha$ implies $\vdash \alpha$.
    Furthermore, since every strict $\Sigma$-sentence is a $\Sigma_{\lhd}$-sentence, 
    it suffices to prove the result for every $\Sigma_{\lhd}$-sentence $\alpha$.
    We proceed by induction on the length $\lambda (\alpha)$.

    Suppose $\lambda(\alpha) = 0$. Then $\alpha$ must be of the form $\sigma \in \tau$ or 
    $\sigma = \tau$ for some constant terms $\sigma, \tau$.
    The result follows directly from the definition of ${\in}$ and ${=}$ in $\mathfrak{S}$,
    i.e. in Definition \ref{def:stdModel}.

    Now assume $\lambda(\alpha) > 0$ and suppose the result holds for all $\Sigma_{\lhd}$-sentences 
    of length less than $\lambda (\alpha)$.
    We consider the following cases:
    \begin{itemize}
        \item $\alpha$ is $\beta \lor \gamma$. By the inductive assumption,
        \begin{equation*}
            \mathfrak{S} \vDash \alpha \rightarrow 
            (\mathfrak{S} \vDash \beta \text{ or } \mathfrak{S} \vDash \gamma) \rightarrow
            (\vdash \beta \text{ or } \vdash \gamma) \rightarrow 
            \vdash \beta \lor \gamma.
        \end{equation*}
        \item $\alpha$ is $\beta \land \gamma$. Analogous to the previous case.
        \item $\alpha$ is $\exists x \beta (x)$.
        If $\mathfrak{S} \vDash \alpha$, then there exists a term $\tau$ such that
        $\mathfrak{S} \vDash \beta(\tau)$. 
        Since $\lambda(\beta(\tau)) = \lambda(\beta) < \lambda(\alpha)$, it follows from
        the inductive assumption that $\vdash \beta(\tau)$, 
        and hence $\vdash \exists x \beta (x)$.
        \item $\alpha$ is $\forall (x \in \tau) \beta(x)$, where $\tau$ is a term that does not
        contain $x$. Since there are no free variables in $\alpha$, $\tau$ must be a constant term.
        By Lemma \ref{lem:C.exists_finset_shorter_and_mem_iff_iSup}, there are 
        $\tau_1, \ldots, \tau_m \in \mathbb{C}$ such that
        $$
        \vdash \forall (x \in \tau) \beta(x) \leftrightarrow
        \beta(\tau_1) \land \ldots \land \beta(\tau_m).
        $$
        Since $\lambda(\beta(\tau_j)) = \lambda(\beta(x)) < \lambda(\alpha)$, it follows 
        from the inductive assumption that
        \begin{equation*}
            \begin{split}
                & \mathfrak{S} \vDash \alpha \rightarrow 
                \mathfrak{S} \vDash \beta(\tau_1) \text{ and } \ldots \text{ and } 
                \mathfrak{S} \vDash \beta(\tau_m) \rightarrow \\
                &\vdash \beta(\tau_1) \text{ and } \ldots \text{ and } \vdash \beta(\tau_m) 
                \rightarrow
                \vdash \beta(\tau_1) \land \ldots \land \beta(\tau_m) \rightarrow
                \vdash \forall (x \in \tau) \beta(x).
            \end{split}
        \end{equation*}
    \end{itemize}
\end{proof}
