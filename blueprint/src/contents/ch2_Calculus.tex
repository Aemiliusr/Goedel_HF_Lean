\chapter{The Logical Calculus of HF}

Recall from Chapter 1 that 
the first-order theory of HF is determined by the first-order language $\mathcal{L}$ 
and three axioms — Axiom \ref{ax:empty}, Axiom \ref{ax:enlarge} and Axiom \ref{ax:induction} — 
from now on referred to as the axioms of HF.
Formally, this explanation is incomplete, 
as we have not yet specified the logical axioms and deduction rules of HF.
In this chapter we will introduce the logiccal axioms and the deduction rules
of HF, which completes the description of the formal system of HF.

Need text on not having used logical calulus in Chapter 1, but...
An $\mathcal{L}$-formula $\varphi$ is a \textit{theorem of HF} (notation $\vdash \varphi$) if there is a 
sequence of $\mathcal{L}$-formulas, terminating with $\varphi$, where each formula in the sequence
is either an axiom of HF, a logical axiom or is derived from one or two formulas in the 
sequence by a deduction rule.

\section{Logical axioms}

The \textit{logical axioms} of HF are the following formulas.

\begin{definition}[Sentential (Boolean) Axioms]
    \label{ax:Bool.Axiom}
    \lean{HF.Bool.Axiom1, HF.Bool.Axiom2, HF.Bool.Axiom3, HF.Bool.Axiom4, HF.Bool.Axiom5}
    \leanok
    For any formulas $\varphi, \psi, \mu$:
    $$
    \begin{aligned}
    \varphi & \rightarrow \varphi, \\
    \varphi & \rightarrow \varphi \lor \psi, \\
    \varphi \lor \varphi & \rightarrow \varphi, \\
    \varphi \lor(\psi \lor \mu) & \rightarrow(\varphi \lor \psi) \lor \mu, \\
    (\varphi \lor \psi) \land(\neg \varphi \lor \mu) & \rightarrow \psi \lor \mu .
    \end{aligned}
    $$
\end{definition}

\begin{definition}[Specialisation axiom]
    \label{ax:Spec.Axiom}
    \lean{HF.Spec.Axiom}
    \leanok
    For any formula $\varphi$ and every $x_i$:
    $$\varphi \rightarrow \exists x_i \varphi.$$
\end{definition}

\begin{definition}[Equality Axioms]
    \label{ax:Equality.Axiom}
    \lean{HF.Equality.Axiom1, HF.Equality.Axiom2, HF.Equality.Axiom3, HF.Equality.Axiom4}
    \leanok
    $$
    \begin{aligned}
        x_1 & = x_1, \\
        \left(x_1=x_2\right) \land \left(x_3=x_4\right) & 
        \rightarrow \left[\left(x_1=x_3\right) \rightarrow \left(x_2=x_4\right)\right], \\
        \left(x_1=x_2\right) \land \left(x_3=x_4\right) & 
        \rightarrow \left[\left(x_1 \in x_3\right) \rightarrow \left(x_2 \in x_4\right)\right], \\
        \left(x_1=x_2\right) \land \left(x_3=x_4\right) & 
        \rightarrow \left[x_1 \lhd x_3 = x_2 \lhd x_4 \right]. 
    \end{aligned}
    $$
\end{definition}

\section{Deduction rules}
\begin{definition}[Deduction rules]
    \label{ded:MP+Subst+ExIntro}
    \lean{HF.prf.MP}
    For any formulas $\varphi, \psi$, the \textit{deduction rules} of HF are as follows:
    \begin{description}
        \item \textit{Modus Ponens}: 
        From $\varphi$ and $\varphi \rightarrow \psi$, deduce $\psi$.
        \item \textit{Substitution}:
        From $\varphi$, deduce $\varphi[x_i/\tau]$ for any term $\tau$ that is substitutable for
        $x_i$ in $\varphi$.
        \item \textit{$\exists$-introduction}:
        From $\varphi \rightarrow \psi$, deduce $\exists x_i \varphi \rightarrow \psi$ provided
        $x_i$ does not occur freely in $\psi$.
    \end{description}
\end{definition}

