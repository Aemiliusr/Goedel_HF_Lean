\chapter*{Introduction}
\addcontentsline{toc}{chapter}{Introduction} 

Gödel’s incompleteness theorems are two of the most significant findings in mathematical logic. 
These theorems, proved by \cite{godel1931formal},
address the boundaries of what can be proven within formal systems. 
The first incompleteness theorem asserts that in any consistent formal system, 
capable of performing basic arithmetic, 
there will always exist certain statements in the system's language that can neither be proved
nor disproved within that system. 
The second incompleteness theorem extends this idea, 
demonstrating that if such a formal system is consistent, it cannot prove its own consistency.

When Gödel's incompleteness theorems were first published, 
they generated a considerable amount of surprise within the mathematical community. 
To understand why, it is important to consider their impact on Hilbert's program, 
a foundational project proposed by David Hilbert in the early 1920s. 
Hilbert aimed to provide a secure basis for all of mathematics by establishing a 
finite, complete and consistent set of axioms from which all mathematical truths could be derived. 
This initiative also led to the emergence of formal languages and first-order logic. 
However, Gödel's theorems demonstrated that such a complete and consistent set of axioms 
was unattainable.

As previously mentioned, Gödel's incompleteness theorems apply to any formal system 
that is consistent and complex enough to describe the basic arithmetic of natural numbers. 
A formal system is a deductive framework made up of a specific set of axioms and deduction rules, 
which govern how new theorems can be logically derived from those axioms.
The incompleteness theorems focus on what can be formally proven within these systems, that is,
solely using the axioms and the deduction rules of the system.
This is a crucial distinction, as it does not concern what can be "informally" proven
in an absolute sense.
For instance, one should not confuse the first incompleteness theorems with the idea that
there are mathematical truths that cannot be proven at all.

First-order theories form a class of formal systems that are particularly relevant to 
Gödel's theorems.
PA is an example of a first-order theory, 
where all variables are meant to represent natural numbers. 
In contrast, systems like the first-order theory of hereditarily finite sets — 
the theory we will focus on — are richer than PA in the sense that 
only certain formulas correspond to statements about natural numbers. 

The first-order theory of hereditarily finite sets (HF) is a formal system where all variables are 
meant to represent sets. It is specified by a first-order language where the non-logical symbols
consist of one binary relation symbol, one binary function symbol, and one constant symbol.
Furthermore, the non-logical axioms consist of two axioms, which essentially define what the 
function symbols represent, and one induction scheme.
We will show that all Zermelo-Fraenkel set theory (ZF) axioms, except for the axiom of infinity,
can be derived from the axioms of HF.
The axiomatic system of ZF, together with the axiom of choice, 
is the common foundation of mathematics.
In fact, for most areas of mathematics, ZF is more than sufficient for deriving the desired results.
Thus, proving Gödel's first incompleteness theorem for HF is significant in the sense that
it shows that even a system that is rich enough to derive a substantial portion of mathematics 
is still incomplete, if it were to be consistent.

Furthermore, we choose HF for proving Gödel's incompleteness theorems as 
the development of a meta-theory — a crucial step in the proof of the 
first incompleteness theorem — is more straightforward in HF than in PA.
As described by \cite{swierczkowski2003finite}, terms and formulas can be encoded directly 
using standard set-theoretical objects, without needing to rely on prime factorisation or 
the Chinese remainder theorem.
To illustrate the ease of the encoding, it is worth mentioning that 
even encoding proofs does not require any additional machinery, whereas for PA there is no natural
way to encode a sequence of codes as one code.
For HF, a sequence of codes of formulas that corresponds to a proof, 
i.e. a sequence of finite sets, is again a finite set and thus can directly serve as a code.
Additionally, \cite{swierczkowski2003finite} proves that HF and PA are definitionally equivalent,
i.e. each of HF and PA is an extension of the other. More specifically,
if one adds the binary relation symbol and the binary function symbol of HF to PA, 
then the axioms of HF can be proved in PA. The analogous converse is also true.
Therefore, it seems that proving Gödel's incompleteness theorems for one of these two theories 
holds the same significance as proving it for the other.

In this thesis, we aim to formalise Gödel's first incompleteness theorem for the first-order theory of
hereditarily finite sets in the Lean theorem prover. 
That is, we present all the results required to prove the first
incompleteness theorem, and formalise as much of the proof as possible within the constraints of 
time. We adopt a similar approach to \cite{paulson2014machine}, where the author formalises the 
developments of \cite{swierczkowski2003finite} in Isabelle.
Formalising the first incompleteness theorem directly leads to the second incompleteness theorem
by readily following the remaining sections of \cite{swierczkowski2003finite}.
At this moment, we are not aware of any existing formalisation of both Gödel's incompleteness 
theorems in Lean.

Before we make the distinction between what has been formalised and what has not,
we will first provide an overview of the proof of Gödel's first incompleteness theorem for HF.
The proof by contradiction is divided into three parts.
First, all symbols, terms, and formulas in the language are encoded as constant terms within the
language itself through a one-to-one mapping.
This mapping leads to the creation of a meta-theory, 
which allows the formal system to make statements about its own content. 
In particular, the encoding lets us analyse properties of formulas — such as whether a formula 
is a theorem of HF — by studying the properties of their codes.
This brings us to the second part of the proof, where the provability formula is constructed.
The provability formula is a condition applicable to the code of a formula $\varphi$ 
in the language of HF that is satisfied if and only if $\varphi$ is a theorem of HF.
So, in essence, the provability formula states 'formula $\varphi$ is provable within the system'.
Finally, using both the coding system and the provability formula, 
the third part of the proof involves constructing a formula that is self-referential and 
essentially states 'I am not provable within the system'. This is established through 
Gödel's diagonal lemma, which finds its roots in Cantor's diagonal argument.
It is then straightforward to observe this formula can neither be proved nor disproved.

Before coming to the formalisation of the coding system and the provability formula, naturally,
as a first step, the first-order theory of hereditarily finite sets has to be formalised.
It includes defining the language of HF, the axioms of HF, the notion of a formal proof in HF,
and the notion of a model of HF — this has been completed by drawing on the work of
\cite{han2020formal}.
In the first place, the axioms of HF generate a vast amount of auxiliary results required for 
formalising the coding system, the provability formula and Gödel's diagonal lemma. 
This part of the work has been mostly completed, 
except for the section dealing with the rank function.
Furthermore, constructing the provability formula requires specifying the standard model of HF.
In Lean, the standard model of HF has been defined, but two proofs in the associated chapter
are still missing.
The coding system has been fully formalised, but the proof of the encoding being one-to-one
is missing. However, the proof of the first incompleteness theorem seems to be independent of the
latter.
The main body of the construction of the provability formula has not been formalised yet,
and the same holds for Gödel's diagonal lemma.
However, using unproven statements regarding the provability formula and Gödel's diagonal lemma, 
the proof of the first incompleteness theorem has been completed. 

For a detailed overview of what has been formalised and what has not, 
the reader is referred to the dependency graph.
Furthermore, 
Chapter 1 provides a detailed introduction to the first-order theory of hereditarily finite sets,
including both syntactical and semantic notions.
Chapter 2 presents the results preliminary to the main steps of the proof of Gödel's first
incompleteness theorem.
Chapters 3 and 4 define the standard model of HF and the coding system respectively.
Finally, Chapter 5 constructs the provability formula and Chapter 6
proves Gödel's diagonal lemma and his first incompleteness theorem.

Note that the presentation of mathematical results in this work closely follows the presentation of  
\cite{swierczkowski2003finite}, because the purpose of this work is to formalise 
the contents of the latter exactly.
The detailed presentation of these results is necessary, 
as the blueprint serves as a guide for the formalisation process, 
contributing to possible future efforts completing this process of proving
Gödel's incompleteness theorems in Lean.

\section*{Acknowledgements}

Primarily, I would like to express my gratitude to my supervisor, Professor Kevin Buzzard, 
for his continuous support, guidance, and enthusiasm for the Lean theorem prover, 
which have been invaluable to this project.
I also thank Jujian Zhang for his collaboration on parts of the formalisation process and 
Professor David Evans for sharing his expertise in mathematical logic. 
Finally, I extend my thanks to the Lean community — especially Floris van Doorn, Sky Wilshaw and 
Bhavik Mehta — for their help and support.


