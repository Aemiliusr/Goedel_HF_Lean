\chapter{The First-Order Theory of Hereditarily Finite Sets}

Text about formal system to first-order theory, and first-order logic in Lean.

\section{Language and axioms}

The \textit{first-order theory of HF} is determined by the first-order language of HF, 
the (non-logical) axioms of HF, the logical axioms, and the deduction rules.

\begin{definition}[The language of HF]
    \label{def:Lang}
    \lean{HF.Lang}
    \leanok
    The \textit{first-order language of HF}, denoted by $\mathcal{L}$, 
    has an alphabet of \textit{symbols} given by the 
    logical constants $=$, $\lor$, $\neg$, $\exists$, the variables denoted by 
    lower case Latin letters, one binary relation symbol $\in$, one binary function symbol $\lhd$, 
    and one constant symbol $\varnothing$.
    The connectives $\land$, $\rightarrow$, $\leftrightarrow$, and $\forall$ are considered 
    abbreviations.
\end{definition}

It follows that any \textit{term} of $\mathcal{L}$ arises in the following way:
any variable is a term, any constant symbol is a term, and if $\tau_1, \tau_2$ are terms, then so is
$\tau_1 \lhd \tau_2$.
Furthermore, any \textit{$\mathcal{L}$-formula} arises in the following way:
any atomic formula is an $\mathcal{L}$-formula, and if $\phi, \psi$ are $\mathcal{L}$-formulas, 
then so are $\neg \phi$, $\phi \lor \psi$, and $\exists x \phi$.
Again, $\phi \land \psi$, $\phi \rightarrow \psi$, $\phi \leftrightarrow \psi$ and
$\forall x \phi$ are treated as definitions.

The axioms of HF consist of two axioms and one scheme. 
Informally, the first axiom defines $\varnothing$ as the
empty set, the second axiom defines $\lhd$ as the operation of adjoining an element to a set,
and the axiom scheme describes proofs by induction.

\begin{definition}[The axioms of HF]
    \label{def:Axioms}
    \lean{HF.Axiom1, HF.Axiom2, HF.Axiom3}
    \uses{def:Lang}
    \leanok
    The \textit{axioms of HF} are the following $\mathcal{L}$-formulas:
    \begin{enumerate}
        \item $z=\varnothing \leftrightarrow \forall x(x \notin z)$.
        \item $z=x \lhd y \leftrightarrow \forall u(u \in z \leftrightarrow u \in x \lor u=y)$.
        \item $(\alpha(\varnothing) \land \forall x \forall y[\alpha(x) \land \alpha(y) \rightarrow 
        \alpha(x \lhd y)]) \rightarrow \forall x \alpha(x)$,
    \end{enumerate}
    where $\alpha$ is any $\mathcal{L}$-formula which contains a 
    distinguished freely occurring variable $z$ such that $x$ and $y$ are substitutable for $z$.
\end{definition}

\section{Logical calculus}

This section recalls the logical axioms and deduction rules for a first-order theory, 
as applied to HF. 
It can be readily shown that the \textit{logical calculus of HF}, as described below, 
is equivalent to the logical calculus of a first-order theory in \cite{shoenfield1967mathematical} 
(when applied to HF), in the sense that they produce the same theorems.

The \textit{logical axioms} are the following $\mathcal{L}$-formulas:

\begin{definition}[Sentential (Boolean) axioms]
    \label{def:Bool.Axioms}
    \lean{HF.Bool.Axiom1, HF.Bool.Axiom2, HF.Bool.Axiom3, HF.Bool.Axiom4, HF.Bool.Axiom5}
    \uses{def:Lang}
    \leanok
    For any $\mathcal{L}$-formulas $\varphi, \psi, \mu$:
    $$
    \begin{aligned}
    \varphi & \rightarrow \varphi, \\
    \varphi & \rightarrow \varphi \lor \psi, \\
    \varphi \lor \varphi & \rightarrow \varphi, \\
    \varphi \lor(\psi \lor \mu) & \rightarrow(\varphi \lor \psi) \lor \mu, \\
    (\varphi \lor \psi) \land(\neg \varphi \lor \mu) & \rightarrow \psi \lor \mu .
    \end{aligned}
    $$
\end{definition}

\begin{definition}[Specialisation axiom]
    \label{def:Spec.Axiom}
    \lean{HF.Spec.Axiom}
    \uses{def:Lang}
    \leanok
    For any $\mathcal{L}$-formula $\varphi$ and every $x_i$:
    $$\varphi \rightarrow \exists x_i \varphi.$$
\end{definition}

\begin{definition}[Equality axioms]
    \label{def:Equality.Axioms}
    \lean{HF.Equality.Axiom1, HF.Equality.Axiom2, HF.Equality.Axiom3, HF.Equality.Axiom4}
    \uses{def:Lang}
    \leanok
    $$
    \begin{aligned}
        x_1 & = x_1, \\
        \left(x_1=x_2\right) \land \left(x_3=x_4\right) & 
        \rightarrow \left[\left(x_1=x_3\right) \rightarrow \left(x_2=x_4\right)\right], \\
        \left(x_1=x_2\right) \land \left(x_3=x_4\right) & 
        \rightarrow \left[\left(x_1 \in x_3\right) \rightarrow \left(x_2 \in x_4\right)\right], \\
        \left(x_1=x_2\right) \land \left(x_3=x_4\right) & 
        \rightarrow \left[x_1 \lhd x_3 = x_2 \lhd x_4 \right]. 
    \end{aligned}
    $$
\end{definition}

The \textit{deduction rules} are as follows:

\begin{definition}[Deduction rules]
    \label{def:prf.MP+prf.Subst+prf.ExIntro}
    \lean{HF.prf.MP}
    \uses{def:Lang}
    For any $\mathcal{L}$-formulas $\varphi, \psi$:
    \begin{description}
        \item \textit{Modus Ponens}: 
        From $\varphi$ and $\varphi \rightarrow \psi$, deduce $\psi$.
        \item \textit{Substitution}:
        From $\varphi$, deduce $\varphi[x_i/\tau]$ for any term $\tau$ that is substitutable for
        $x_i$ in $\varphi$.
        \item \textit{$\exists$-introduction}:
        From $\varphi \rightarrow \psi$, deduce $\exists x_i \varphi \rightarrow \psi$ provided
        $x_i$ does not occur freely in $\psi$.
    \end{description}
\end{definition}

These descriptions naturally lead to the notion of formal proof.

\begin{definition}[Theorem of HF]
    \label{def:prf}
    \lean{HF.prf}
    \uses{def:Lang, def:Axioms, def:Bool.Axioms, def:Spec.Axiom, 
        def:Equality.Axioms, def:prf.MP+prf.Subst+prf.ExIntro}
    An $\mathcal{L}$-formula is a \textit{theorem of HF}, denoted by $\vdash \phi$, 
    if there is a finite sequence of $\mathcal{L}$-formulas — each of which is an axiom of HF, 
    a logical axiom, or is deduced from previous formulas in the sequence by a deduction rule —
    terminating with $\phi$.
\end{definition}

Following the approach outlined by \cite{paulson2014machine}, 
the syntactic proof system of HF is implemented in Lean as an inductively defined predicate. 
In this setup, a proof within this system, resulting in a theorem of HF, 
involves demonstrating in Lean’s native logic that 
the proof predicate applied to the relevant $\mathcal{L}$-formula evaluates to true.
However, in practice, working directly within this formal calculus is cumbersome. 
\cite{paulson2014machine} experienced that even the simplest formal proofs can become extremely 
tedious and require hundreds of lines of code.
To streamline the process of obtaining theorems of HF in Lean, 
we aim to convert semantic proofs (i.e., using only Lean's native logic and not the proof predicate) 
into proofs within the HF syntactic system. 
This is possible by applying Gödel's completeness theorem, which applies to any first-order theory.

\section{Semantics}

Need text on use of semantic notions.

\begin{definition}
    \label{def:models+Model+valid}
    \lean{HF.models, HF.Model, HF.valid}
    \uses{def:Lang, def:Axioms}
    \leanok
    Suppose $\mathcal{S}$ is an $\mathcal{L}$-structure and $\phi$ is an $\mathcal{L}$-formula.
    If every valuation in $\mathcal{S}$ satisfies $\phi$, 
    say $\mathcal{S}$ is a \textit{model} of $\phi$, denoted by $\mathcal{S} \vDash \phi$.

    If $\mathcal{S}$ is a model of every axiom of HF, say $\mathcal{S}$ is a \textit{model of HF}.
    
    An $\mathcal{L}$-formula $\phi$ is called \textit{valid in HF} if 
    every model of HF is a model of $\phi$, denoted by $\vDash \phi$.
\end{definition}

For a proof of the following result for an arbitrary first-order theory, 
the reader is referred to  \cite{shoenfield1967mathematical}.

\begin{theorem}[Soundness and completeness]
    \label{thm:completeness}
    \lean{HF.completeness}
    \uses{def:Lang, def:prf, def:models+Model+valid}
    \leanok
    Suppose $\phi$ is an $\mathcal{L}$-formula.
    Then $\vdash \phi$ if and only if $\vDash \phi$.
\end{theorem}

It other words, the theorems of HF are precisely the $\mathcal{L}$-formulas that are valid in HF.
Thus, to prove $\vdash \phi$ for any $\mathcal{L}$-formula $\phi$, it suffices to show that for an 
arbitrary model of HF, $\mathcal{S}$, we have $\mathcal{S} \vDash \phi$.
This is the conversion process we previously discussed.

In Lean 3, the completeness theorem for first-order logic was proven as part of the Flypitch project
\cite{han2020formal}.
However, there is no proof available in the math library of Lean 4.
For now, we make use of soundness and completeness by accepting the statement 
using the {\ttfamily \small sorry} identifier. 